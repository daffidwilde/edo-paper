\documentclass[11pt]{article}

\usepackage[margin=2cm, includefoot, footskip=30pt]{geometry}
\usepackage{hyperref}
\usepackage{tcolorbox}

\title{Response to reviewers for manuscript APIN-D-19-01616}
\author{Henry Wilde, Vincent Knight, Jonathan Gillard}
\date{}

\begin{document}

\maketitle%

Included in this document are the responses of these authors to the comments
made by the reviewers in regard to the manuscript submitted to Applied
Intelligence as ``Evolutionary Dataset Optimisation: learning algorithm quality
through evolution''.

\section*{Reviewer 1}

\begin{tcolorbox}
\begin{verbatim}
    I think the subject and object of this paper is very ambiguous in the
    introduction even though authors well described the previous work, I hope
    the author should consider why this paper is necessary to the read.
\end{verbatim}
\end{tcolorbox}

The introduction has been revised to be more direct in its summary of the work,
and its motivation has been expanded on. In particular, we have laboured the
point even more that the proposed method belongs to a new paradigm in which no
methods are currently published. We have expanded on the issues we raise with
the current paradigm and provide further references to support our statements
(see paragraph 4 of the introduction).\\

\begin{tcolorbox}
\begin{verbatim}
    The language needs to be revised by native speaker research.
\end{verbatim}
\end{tcolorbox}

The manuscript was written by three native English speakers but we have made
sure that any lingering spelling mistakes or grammatical issues have been
addressed. If there are any specific language errors that the reviewer can point
out they will be addressed.\\

\begin{tcolorbox}
\begin{verbatim}
    make the ABSTRACT as a single paragraph and make sure do present your work
    in clearer.
\end{verbatim}
\end{tcolorbox}

In this draft of the paper, the abstract has been reduced to a single
paragraph.\\

\begin{tcolorbox}
\begin{verbatim}
    You should improve the ABSTRACT.(rewrite the abstract to reflect the main
    idea and it's results, without any not suitable details) in the ABSTRACT
    alongside with the obtained results (the results you got it and what is the
    situation of your results in comparison with other published methods).
    Mentioned to the benchmarks which have been used in this paper.
\end{verbatim}
\end{tcolorbox}

We feel that the main concept of the method is summarised well in the abstract,
including its motivation. As is discussed throughout this response, and in the
article itself, classical results are not included in this work and no
benchmarks are used hence those points are omitted from the abstract. However,
we have revised the final sentence of the abstract to more accurately describe
the case study provided at the end of the article. In particular, we state that
`a number of known [favourable] properties' are identified by the proposed
method for \(k\)-means and DBSCAN so that there is no ambiguity in the results
displayed in the paper.\\

\begin{tcolorbox}
\begin{verbatim}
    in figure 1, there are question marks, you did not explain that. why?
\end{verbatim}
\end{tcolorbox}

The choice for question marks was to symbolise that a series of questions may be
asked of the data. This has now been done explicitly by changing the annotation
in the figure to be `Asking questions of the data'. In addition to this, the
same phrase has been used in the text when describing the process that it
describes in Figure 1.\\

\begin{tcolorbox}
\begin{verbatim}
    Add in the and of section 1, add a new paragraph that presents the
    organization of the paper.
\end{verbatim}
\end{tcolorbox}

As can be seen at the end of the `Introduction' section, there is now a summary
of the organisation of the paper by section.\\

\begin{tcolorbox}
\begin{verbatim}
    Unify the symbols such as Figure or Fig.
\end{verbatim}
\end{tcolorbox}

This was a minor oversight that has now been amended. Thank you for bringing
this to our attention.\\

\begin{tcolorbox}
\begin{verbatim}
    The related works section is not provided. I suggest to increase the number
    of studies and add anew discussion there to show the advantage,
    disadvantage, and weakness of the studied works. Authors should discuss the
    literature review more deep and clearly.
\end{verbatim}
\end{tcolorbox}

We have expanded the size of the bibliography in the paper, adding references to
enrich the introduction and algorithm description. These are as follows:
\begin{itemize}
    \item Abualigah, L.M., Khader, A.T., Hanandeh, E.S.: A combination of
        objective functions and hybrid krill herd algorithm for text document
        clustering analysis. Engineering Applications of Artificial Intelligence
        73, 111 \-- 125 (2018). DOI 10.1016/j.engappai.2018.05.003
    \item Abualigah, L.M., Khader,A.T., Hanandeh,E.S.: Hybrid clustering
        analysis using improved krill herd algorithm. Applied Intelligence
        48(11), 4047 \-- 4071 (2018). DOI 10.1007/s10489-018-1190-6
    \item Campos, G., Zimek, A., Sander, J., Campello, R., Micenkov, B.,
        Schubert, E., Assent, I., Houle, M.: On the evaluation of unsupervised
        outlier detection: measures, datasets, and an empirical study. Data
        Mining and Knowledge Discovery 30(4), 891 \-- 927 (2016). DOI
        10.1007/s10618-015-0444-8
    \item Chen, Y., Elliot, M., Smith, D.: The application of genetic algorithms
        to data synthesis: A comparison of three crossover methods. In: Privacy
        in Statistical Databases, pp. 160 \-- 171. Springer International
        Publishing (2018). DOI 10.1007/ 978-3-319-99771-1 11
    \item Dau, H.A., Keogh, E., Kamgar, K., Yeh, C.C.M., Zhu, Y., Gharghabi, S.,
        Ratanamahatana, C.A., Yanping, Hu, B., Begum, N., Bagnall, A., Mueen,
        A., Batista, G., Hexagon-ML: The UCR time series classification archive
        (2018). \url{https://www.cs.ucr.edu/~eamonn/time_series_data_2018/}
    \item Liu, L., Cheng, L., Liu, Y., Jia, Y., Rosenblum, D.: Recognizing
        complex activities by a probabilistic interval-based model (2016)
    \item Olson, R.S., La Cava, W., Orzechowski, P., Urbanowicz, R.J., Moore,
        J.H.: PMLB: a large benchmark suite for machine learning evaluation and
        comparison. BioData Mining 10(1), 36 (2017). DOI
        10.1186/s13040-017-0154-4
    \item Vikhar, P.A.: Evolutionary algorithms: A critical review and its
        future prospects. In: 2016 International Conference on Global Trends in
        Signal Processing, Information Computing and Communication (ICGTSPICC),
        pp. 261–265 (2016). DOI 10.1109/ICGTSPICC.2016.7955308
    \item Wang, N., Shi, J., Yeung, D.Y., Jia, J.: Understanding and diagnosing
        visual tracking systems (2015). DOI 10.1109/ICCV.2015.355
    \item Zhao, W., Ma, H., He, Q.: Parallel k-means clustering based on
        MapReduce. In: Cloud Computing, pp. 674 \-- 679. Springer Berlin
        Heidelberg (2009). DOI 10.1007/978-3-642-10665-1 71
\end{itemize}

As is discussed throughout this response, the proposed method has no
contemporaries in this paradigm. As such, the close discussion of the studied
works would be not of benefit to the paper as they are not relevant in the
classical sense where their contents is comparable to that of this work. The
purpose of the introduction, really, is to provide the motivation of the paper
of which the proposed method forms only a part. The root of the motivation
(summarised in the numbered list in the introduction) is not novel and is
addressed in a number of the cited works in the introduction. The fact that the
proposed method is an example of a method from this paradigm is discussed in the
later part of the introduction where other methods are addressed.

We refer to the editor if a standalone `related works' section is required since
it is not included as part of the Submission Guidelines. The guidelines are
available at the following link:
\url{https://www.springer.com/journal/10489/submission-guidelines?IFA}.\\

\begin{tcolorbox}
\begin{verbatim}
    Most references in the list of reference are very old, therefore, must
    update the reference list to contain articles related off at least five
    years and indexing in ISI and Scopus Database, in general update that list
    by the following reference related to predictions: 1-Abualigah, L. M. Q.
    (2019). Feature Selection and Enhanced Krill Herd Algorithm for Text
    Document Clustering. Studies in Computational Intelligence. 2-Abualigah, L.
    M. Q., & Hanandeh, E. S. (2015). Applying genetic algorithms to information
    retrieval using vector space model. International Journal of Computer
    Science, Engineering and Applications, 5(1), 19. 3-Abualigah, L. M., &
    Khader, A. T. (2017). Unsupervised text feature selection technique based on
    hybrid particle swarm optimization algorithm with genetic operators for the
    text clustering. The Journal of Supercomputing, 73(11), 4773-4795.
    4-Abualigah, L. M., Khader, A. T., & Hanandeh, E. S. (2018). Hybrid
    clustering analysis using improved krill herd algorithm. Applied
    Intelligence. 5-Abualigah, L. M., Khader, A. T., & Hanandeh, E. S. (2018). A
    Combination of Objective Functions and Hybrid Krill Herd Algorithm for Text
    Document Clustering Analysis. Engineering Applications of Artificial
    Intelligence. 6-Abualigah, L. M., Khader, A. T., & Hanandeh, E. S. (2017). A
    new feature selection method to improve the document clustering using
    particle swarm optimization algorithm. Journal of Computational Science.
\end{verbatim}
\end{tcolorbox}

While some of the references are older than 5 years, this is often due to the
fact that they are seminal papers for particular concepts, algorithms and
software packages. For this reason, they are entirely relevant to the paper and
have been included.

Taking this comment into account, however, we have revised our reference list to
include two of the suggested articles (4 \& 5) and other contemporary works
where appropriate.\\

\begin{tcolorbox}
\begin{verbatim}
    You need to explain clearly your proposed methods epically for the proposed
    method.

    Add a new figure to show the general procedures of the proposed method
\end{verbatim}
\end{tcolorbox}

We believe that the proposed method is explained in great detail with
supplementary diagrams and algorithms for each subprocess; to add an `epic'
explanation to this description would make the manuscript cumbersome in terms of
its length.

In terms of adding a new figure, there are diagrams which describe the concept
and motivation of the work (Figure 1), the general structure of the method
algorithmically (Figure 2), and each subprocess of the method (Figures 3\--6).
It is unclear what kind of figure is being asked for here which has not already
been covered.

We refer to the editor on these points as the description of the method has been
a point of pride for the authors that has been specifically commended when the
work has been presented.\\

\begin{tcolorbox}
\begin{verbatim}
    For the experimental results, it will be good to present a statistical test
    in the comparison of the results with other published methods. This can help
    to support the claim on improved results obtained with the selection methods
    studied.
\end{verbatim}
\end{tcolorbox}

As is discussed in the introduction, the proposed method comes from a novel
paradigm in which there are no `other published methods' to compare it with. In
this paradigm, the objective of the method is to generate data rather than to
complete some comparable task X where one would be able to do some analysis like
`method A on X vs.\ method B on X'. As this is not possible, there is no
sensible grounding to use a statistical test in this case.\\

\begin{tcolorbox}
\begin{verbatim}
    What are the pros and cons of the proposed method? Please respond to this
    question in the article text.
\end{verbatim}
\end{tcolorbox}

The advantages and disadvantages of the proposed method are now discussed more
explicitly in the conclusion of the paper. In particular, issues around the
total coverage, premature termination and simplistic solutions are discussed
with reference to Figure 15.

\begin{tcolorbox}
\begin{verbatim}
    in this paper, as you claimed, benchmark datasets are proposed. This is not
    clear in the paper
\end{verbatim}
\end{tcolorbox}

This paper does not propose benchmark datasets. Where benchmark datasets are
mentioned, it is to motivate the proposed paradigm and method, and not for
comparison. Within this discussion, references are provided to articles that use
benchmark datasets and to compilations of benchmark datasets themselves. We have
clarified this in the text.

%%%%%%%%%%%%%%%%%%%%%

\section*{Reviewer 2}

\begin{tcolorbox}
\begin{verbatim}
    The key concept behind the work is the evolutionary generation of datasets
    in order to understand the limitations and capabilities of ML algorithms.
    Thus, for this to be effective, you need to demonstrate that your
    representation and genetic operators are capable of permitting evolution to
    generate any possible dataset - or at least the set of datasets that would
    adequately cover the sets corresponding to the ML algorithm under
    investigation. e.g. If K-means needs to be investigated with sets A,B,C,D,E
    (each of which might be a different class of distribution) but your
    evolutionary approach is only capable of generating sets A,C,E, then you may
    gain an incomplete or misleading view of the capabilities of the algorithm.
    It is a big ask perhaps to prove that your method can generate *any* data
    distribution (or is not overly biassed towards the generation of some
    compared to others because of the representation or operators) and I'm sure
    later versions would be able to improve in these areas, but this issue is so
    important for the approach to be viable, that I think you should provide
    some evidence in the paper that the method can produce adequate coverage.
\end{verbatim}
\end{tcolorbox}

Thank you for this extensive comment. We agree that it may be an overstatement
to say that our method can produce \textbf{any} dataset so we have revised our
language throughout the text. More importantly, though, we have addressed the
ability for our method to produce `adequate' coverage. In Figure 15, there is a
scatter plot showing the distribution of all individuals from every 50
generations using the first of the DBSCAN examples. The main conclusions of this
plot are discussed in the conclusion of the article but may summarised as: (a)
the method has covered a very substantial part of the search space, and (b) the
method is indeed able to identify some preferable behaviour rather than
persisting a full random cloud over the unit square.

\begin{tcolorbox}
\begin{verbatim}
    The second concern is the way evolutionary algorithms tend to find the
    easiest way out - so how can you stop the EA from evolving the simplest,
    easiest datasets for each ML algorithm (or the most difficult datasets if
    you reverse the fitness), instead of exploring the range of possibilities?
    (One option is to move more into curiosity-driven search, perhaps?)
\end{verbatim}
\end{tcolorbox}

This comment has been addressed by quite a substantial expansion of the
conclusion. In the last few paragraphs, a meta-learning methodology is suggested
for how this pitfall may be avoided within the scope of the proposed method. In
this discussion, we address the fact that although the method is unsupervised,
by designing an appropriately sophisticated fitness function a user could adopt
a reinforced (or otherwise semi-supervised) learning behaviour with the proposed
method.

We have also stressed (in the introduction and conclusion) the particular
reasons for choosing an EA in this work, and that this method is just an example
of what may come from further research within this new paradigm.

\end{document}
