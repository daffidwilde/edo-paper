\documentclass[11pt]{article}

\usepackage[margin=1in, includefoot, footskip=30pt]{geometry}

\title{Response to reviewers for manuscript APIN-D-19-01616}
\author{Henry Wilde, Vincent Knight, Jonathan Gillard}
\date{}

\begin{document}

\maketitle%

Included in this document are the responses of these authors to the comments
made by the reviewers in regard to the manuscript submitted to Applied
Intelligence as ``Evolutionary Dataset Optimisation: learning algorithm quality
through evolution''.

\section*{Reviewer 1}

\begin{itemize}

\item
\begin{verbatim}
    I think the subject and object of this paper is very ambiguous in the
    introduction even though authors well described the previous work, I hope
    the author should consider why this paper is necessary to the read.
\end{verbatim}

The introduction has been revised to be more direct in its summary of the work,
and its motivation has been expanded on.

% TODO Reorganise the introduction to be punchier. Address the points raised in
% the introduction.

\item
\begin{verbatim}
    The language needs to be revised by native speaker research.
\end{verbatim}

The manuscript was written by three native English speakers but we have made
sure that any lingering spelling mistakes or grammatical issues have been
addressed. If there are any specific language errors that the reviewer can point
out that would helpful.

\item
\begin{verbatim}
    make the ABSTRACT as a single paragraph and make sure do present your work
    in clearer.
\end{verbatim}

In this draft of the paper, the abstract has been reduced to a single paragraph.

\item
\begin{verbatim}
    You should improve the ABSTRACT.(rewrite the abstract to reflect the main
    idea and it's results, without any not suitable details) in the ABSTRACT
    alongside with the obtained results (the results you got it and what is the
    situation of your results in comparison with other published methods).
    Mentioned to the benchmarks which have been used in this paper.
\end{verbatim}

We feel that the main concept of the method is summarised well in the abstract,
including its motivation. As is discussed throughout this response, and in the
article itself, classical results are not included in this work and no
benchmarks are used hence those points being omitted from the abstract. However,
we have revised the final sentence of the abstract to more accurately describe
the case study provided at the end of the article. In particular, we state that
`a number of known [favourable] properties' are identified by the proposed
method for \(k\)-means.

\item
\begin{verbatim}
    in figure 1, there are question marks, you did not explain that. why?
\end{verbatim}

The choice for question marks was to symbolise that a series of questions may be
asked of the data. This has now been done explicitly by changing the annotation
in the figure to be `Asking questions of the data'.

\item
\begin{verbatim}
    Add in the and of section 1, add a new paragraph that presents the
    organization of the paper.
\end{verbatim}

As can be seen at the end of the `Introduction' section, there is now a summary
of the organisation of the paper by section.

\item
\begin{verbatim}
    Unify the symbols such as Figure or Fig.
\end{verbatim}

This was a minor oversight that has now been amended. Thank you for bringing
this to our attention.

\item
\begin{verbatim}
    The related works section is not provided. I suggest to increase the number
    of studies and add anew discussion there to show the advantage,
    disadvantage, and weakness of the studied works. Authors should discuss the
    literature review more deep and clearly.
\end{verbatim}

% TODO Expand on existing literature review of other methods and make clear that
% the proposed method is largely novel in its motivation. This is the reason
% there is no dedicated literature review section, and that is not required by
% the journal.

% We refer to the editor if section required...

\item
\begin{verbatim}
    Most references in the list of reference are very old, therefore, must
    update the reference list to contain articles related off at least five
    years and indexing in ISI and Scopus Database, in general update that list
    by the following reference related to predictions: 1-Abualigah, L. M. Q.
    (2019). Feature Selection and Enhanced Krill Herd Algorithm for Text
    Document Clustering. Studies in Computational Intelligence. 2-Abualigah, L.
    M. Q., & Hanandeh, E. S. (2015). Applying genetic algorithms to information
    retrieval using vector space model. International Journal of Computer
    Science, Engineering and Applications, 5(1), 19. 3-Abualigah, L. M., &
    Khader, A. T. (2017). Unsupervised text feature selection technique based on
    hybrid particle swarm optimization algorithm with genetic operators for the
    text clustering. The Journal of Supercomputing, 73(11), 4773-4795.
    4-Abualigah, L. M., Khader, A. T., & Hanandeh, E. S. (2018). Hybrid
    clustering analysis using improved krill herd algorithm. Applied
    Intelligence. 5-Abualigah, L. M., Khader, A. T., & Hanandeh, E. S. (2018). A
    Combination of Objective Functions and Hybrid Krill Herd Algorithm for Text
    Document Clustering Analysis. Engineering Applications of Artificial
    Intelligence. 6-Abualigah, L. M., Khader, A. T., & Hanandeh, E. S. (2017). A
    new feature selection method to improve the document clustering using
    particle swarm optimization algorithm. Journal of Computational Science.
\end{verbatim}

While some of the references are older than 5 years, this is often due to the
fact that they are seminal papers for particular concepts, algorithms and
software packages. For this reason, they are entirely relevant to the paper
which is why they have been included.

Taking this comment into account, however, we have revised our reference list to
include two of the suggested articles (4 \&?) and other contemporary works
where appropriate.

% TODO Revise some of the older (pre-2013) references and incorporate some of
% the above. Start with (4).

\item
\begin{verbatim}
    You need to explain clearly your proposed methods epically for the proposed
    method.

    Add a new figure to show the general procedures of the proposed method
\end{verbatim}

We believe that the proposed method is explained in great detail with
supplementary diagrams and algorithms for each subprocess; to add an `epic'
explanation to this description would make the manuscript cumbersome in terms of
its length.

In terms of adding a new figure, there are diagrams which describe the concept
and motivation of the work (Figure 1), the general structure of the method
algorithmically (Figure 2), and each subprocess of the method (Figures 3\--6).
It is unclear what kind of figure is being asked for here which has not already
been covered.

We refer to the editor on these points as the description of the method has been
a point of pride for the authors that has been specifically commended when the
work has been presented in seminars.

% What does epically mean?

\item
\begin{verbatim}
    For the experimental results, it will be good to present a statistical test
    in the comparison of the results with other published methods. This can help
    to support the claim on improved results obtained with the selection methods
    studied.
\end{verbatim}

As is discussed in the introduction, the proposed method comes from a novel
paradigm in which there are no `other published methods' to compare it with. In
this paradigm, the objective of the method is to generate data rather than to
complete some comparable task where traditionally one would be able to do
something like method A vs.\ method B. As such, there is no sensible grounding
to use a statistical test in this case.

\item
\begin{verbatim}
    What are the pros and cons of the proposed method? Please respond to this
    question in the article text.
\end{verbatim}

The advantages and disadvantages of the method are now discussed more explicitly
in the conclusion of the paper. In particular, the issue with using a genetic
algorithm approach is discussed.

% TODO Address this in the conclusion. Positives are discussed but one of the
% biggest limitations may well be that the method cannot produce *any* dataset.

\item
\begin{verbatim}
    in this paper, as you claimed, benchmark datasets are proposed. This is not
    clear in the paper
\end{verbatim}

This paper does not propose benchmark datasets. Where benchmark datasets are
mentioned, it is to motivate the proposed paradigm and method, and not for
comparison. Within this discussion, references are provided to articles that use
benchmark datasets. In addition to these, particular examples have now been
provided as well as references to benchmark dataset surveys to show their
ubiquity.

% TODO Find some benchmark dataset surveys for time series, forecasting, image
% recognition, etc., and include some examples of well-known benchmark datasets.

\end{itemize}

\section*{Reviewer 2}

\begin{itemize}

\item
\begin{verbatim}
    The key concept behind the work is the evolutionary generation of datasets
    in order to understand the limitations and capabilities of ML algorithms.
    Thus, for this to be effective, you need to demonstrate that your
    representation and genetic operators are capable of permitting evolution to
    generate any possible dataset - or at least the set of datasets that would
    adequately cover the sets corresponding to the ML algorithm under
    investigation. e.g. If K-means needs to be investigated with sets A,B,C,D,E
    (each of which might be a different class of distribution) but your
    evolutionary approach is only capable of generating sets A,C,E, then you may
    gain an incomplete or misleading view of the capabilities of the algorithm.
    It is a big ask perhaps to prove that your method can generate *any* data
    distribution (or is not overly biassed towards the generation of some
    compared to others because of the representation or operators) and I'm sure
    later versions would be able to improve in these areas, but this issue is so
    important for the approach to be viable, that I think you should provide
    some evidence in the paper that the method can produce adequate coverage.
\end{verbatim}

% TODO Can this be proven mathematically? I think it could be demonstrated using
% a scatter plot of the data with low opacity so hotspots could be identified.
% Maybe colour code stages of the epochs (100s).

% This can be controlled using the objective function. Why rule it out? This
% will require further study.

\item
\begin{verbatim}
    The second concern is the way evolutionary algorithms tend to find the
    easiest way out - so how can you stop the EA from evolving the simplest,
    easiest datasets for each ML algorithm (or the most difficult datasets if
    you reverse the fitness), instead of exploring the range of possibilities?
    (One option is to move more into curiosity-driven search, perhaps?)
\end{verbatim}

% TODO More literature review required to discuss curiosity search and need to
% address the fact that the concept remains the same in that exploration of
% `good' datasets is not really done. The E/GA is merely a tool that allows for
% this whilst maintaining histories of behaviours.

% Not an exhaustive search, rather completely unsupervised.
% Objective function modified to be run sequentially for further insight.

\end{itemize}
\end{document}
