\documentclass[11pt]{article}

\usepackage[margin=1in, includefoot, footskip=30pt]{geometry}

\title{Response to reviewers for manuscript APIN-D-19-01616}
\author{Henry Wilde, Vincent Knight, Jonathan Gillard}
\date{}

\begin{document}

\maketitle%

Included in this document are the responses of these authors to the comments
made by the reviewers in regard to the manuscript submitted to Applied
Intelligence as ``Evolutionary Dataset Optimisation: learning algorithm quality
through evolution''.

\section*{Reviewer 1}

\begin{itemize}

\item
\begin{verbatim}
    I think the subject and object of this paper is very ambiguous in the
    introduction even though authors well described the previous work, I hope
    the author should consider why this paper is necessary to the read.
\end{verbatim}

The introduction has been revised to be more direct in its summary of the paper,
and its motivation has been expanded on.

% TODO Reorganise the introduction to be punchier. Address the points raised in
% the introduction.

\item
\begin{verbatim}
    The language needs to be revised by native speaker research.
\end{verbatim}

The manuscript was written by three native English speakers but we have made
sure that any lingering spelling mistakes or grammatical issues have been
addressed.

\item
\begin{verbatim}
    make the ABSTRACT as a single paragraph and make sure do present your work
    in clearer.
\end{verbatim}

In this draft of the paper, the abstract has been reduced to a single paragraph.

% TODO Reduce the abstract to a single paragraph.

\item
\begin{verbatim}
    You should improve the ABSTRACT.(rewrite the abstract to reflect the main
    idea and it's results, without any not suitable details) in the ABSTRACT
    alongside with the obtained results (the results you got it and what is the
    situation of your results in comparison with other published methods).
    Mentioned to the benchmarks which have been used in this paper.
\end{verbatim}

% TODO Rework the abstract to be punchier and with less waffle. Address in
% comment that the paper itself uses no benchmark datasets.

\item
\begin{verbatim}
    in figure 1, there are question marks, you did not explain that. why?
\end{verbatim}

The choice for question marks was to symbolise that a series of questions may be
asked of the data. This has now been done explicitly.

% TODO include this explicitly.

\item
\begin{verbatim}
    Add in the and of section 1, add a new paragraph that presents the
    organization of the paper.
\end{verbatim}

As can be seen at the end of the `Introduction' section, there is now a summary
of the organisation of the paper.

% TODO Write contents paragraph.

\item
\begin{verbatim}
    Unify the symbols such as Figure or Fig.
\end{verbatim}

This was a minor oversight that has now been amended.

\item
\begin{verbatim}
    The related works section is not provided. I suggest to increase the number
    of studies and add anew discussion there to show the advantage,
    disadvantage, and weakness of the studied works. Authors should discuss the
    literature review more deep and clearly.
\end{verbatim}

% TODO Expand on existing literature review of other methods and make clear that
% the proposed method is largely novel in its motivation. This is the reason
% there is no dedicated literature review section, and that is not required by
% the journal.

\item
\begin{verbatim}
    Most references in the list of reference are very old, therefore, must
    update the reference list to contain articles related off at least five
    years and indexing in ISI and Scopus Database, in general update that list
    by the following reference related to predictions: 1-Abualigah, L. M. Q.
    (2019). Feature Selection and Enhanced Krill Herd Algorithm for Text
    Document Clustering. Studies in Computational Intelligence. 2-Abualigah, L.
    M. Q., & Hanandeh, E. S. (2015). Applying genetic algorithms to information
    retrieval using vector space model. International Journal of Computer
    Science, Engineering and Applications, 5(1), 19. 3-Abualigah, L. M., &
    Khader, A. T. (2017). Unsupervised text feature selection technique based on
    hybrid particle swarm optimization algorithm with genetic operators for the
    text clustering. The Journal of Supercomputing, 73(11), 4773-4795.
    4-Abualigah, L. M., Khader, A. T., & Hanandeh, E. S. (2018). Hybrid
    clustering analysis using improved krill herd algorithm. Applied
    Intelligence. 5-Abualigah, L. M., Khader, A. T., & Hanandeh, E. S. (2018). A
    Combination of Objective Functions and Hybrid Krill Herd Algorithm for Text
    Document Clustering Analysis. Engineering Applications of Artificial
    Intelligence. 6-Abualigah, L. M., Khader, A. T., & Hanandeh, E. S. (2017). A
    new feature selection method to improve the document clustering using
    particle swarm optimization algorithm. Journal of Computational Science.
\end{verbatim}

While some of the references are older than 5 years, this is almost always due
to the fact that they are seminal papers for particular concepts, algorithms and
software packages. Having said that, we have revised our reference list to
include some of the suggested articles and other contemporary works where
appropriate.

% TODO Revise some of the older (pre-2013) references and incorporate some of
% the above.

\item
\begin{verbatim}
    You need to explain clearly your proposed methods epically for the proposed
    method.
\end{verbatim}

We believe that the proposed method is explained in great detail with diagrams
and algorithms for each subprocess; to add an epic explanation to this
description would make the manuscript cumbersome in terms of its length.

\item
\begin{verbatim}
    Add a new figure to show the general procedures of the proposed method
\end{verbatim}

As stated in the text of the article and above, there are diagrams which
describe the concept and motivation of the work (Figure 1), the general
structure of the method algorithmically (Figure 2), and each subprocess of the
method (Figures 3\--6). It is unclear what kind of figure is being asked for
here which has not already been covered.

\item
\begin{verbatim}
    For the experimental results, it will be good to present a statistical test
    in the comparison of the results with other published methods. This can help
    to support the claim on improved results obtained with the selection methods
    studied.
\end{verbatim}

This is an interesting idea but it is unclear what is meant. Certainly, a
statistical test showing the results of the convexity comparison in the final
examples would be beneficial to the article. However, there are no published
methods that do what is being done in the article.

% TODO Speak with Jon about how this could be done. Will likely require a lot of
% computational time on Siren to run more results.

\item
\begin{verbatim}
    What are the pros and cons of the proposed method? Please respond to this
    question in the article text.
\end{verbatim}

The advantages and disadvantages of the method are now discussed more explicitly
in the conclusion of the paper. In particular, the issue with using a genetic
algorithm approach is discussed.

% TODO Address this in the conclusion. Positives are discussed but one of the
% biggest limitations may well be that the method cannot produce *any* dataset.

\item
\begin{verbatim}
    in this paper, as you claimed, benchmark datasets are proposed. This is not
    clear in the paper
\end{verbatim}

Where this is discussed in the manuscript, references are provided to articles
that use benchmark datasets. In addition, however, examples have now been
provided as well as references to benchmark dataset surveys.

% TODO Find some benchmark dataset surveys for time series, forecasting, image
% recognition, etc., and include some examples of well-known benchmark datasets.

\end{itemize}

\section*{Reviewer 2}

\begin{itemize}

\item
\begin{verbatim}
    The key concept behind the work is the evolutionary generation of datasets
    in order to understand the limitations and capabilities of ML algorithms.
    Thus, for this to be effective, you need to demonstrate that your
    representation and genetic operators are capable of permitting evolution to
    generate any possible dataset - or at least the set of datasets that would
    adequately cover the sets corresponding to the ML algorithm under
    investigation. e.g. If K-means needs to be investigated with sets A,B,C,D,E
    (each of which might be a different class of distribution) but your
    evolutionary approach is only capable of generating sets A,C,E, then you may
    gain an incomplete or misleading view of the capabilities of the algorithm.
    It is a big ask perhaps to prove that your method can generate *any* data
    distribution (or is not overly biassed towards the generation of some
    compared to others because of the representation or operators) and I'm sure
    later versions would be able to improve in these areas, but this issue is so
    important for the approach to be viable, that I think you should provide
    some evidence in the paper that the method can produce adequate coverage.
\end{verbatim}

\item
\begin{verbatim}
    The second concern is the way evolutionary algorithms tend to find the
    easiest way out - so how can you stop the EA from evolving the simplest,
    easiest datasets for each ML algorithm (or the most difficult datasets if
    you reverse the fitness), instead of exploring the range of possibilities?
    (One option is to move more into curiosity-driven search, perhaps?)
\end{verbatim}

\end{itemize}
\end{document}
