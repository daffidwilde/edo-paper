\section{Examples}\label{section:examples}

\subsection{\(k\)-means clustering}

The following examples act as a form of validation for EDO, and also highlight
some of the nuances in its use. The examples will be focused around the
clustering of data and, in particular, the \(k\)-means (Lloyd's) algorithm.
Clustering was chosen as it is a well-understood problem that is easily
accessible \-- especially when restricted to two dimensions. The \(k\)-means
algorithm is an iterative, centroid-based method that aims to minimise the
`inertia' of the current partition, \(Z = \left\{Z_1, \ldots, Z_k\right\}\),
of some dataset \(X\):
\begin{equation}
    I(Z, X) := \frac{1}{|X|} \sum_{j=1}^{k} \sum_{x \in Z_j} {d(x, z_j)}^2
    \label{eq:inertia}
\end{equation}

A full statement of the algorithm to minimise~(\ref{eq:inertia}) is given
in~\ref{app:kmeans}. 

This inertia function is taken as the objective of the \(k\)-means algorithm,
and is used for evaluating the final clustering. This is particularly true when
the algorithm is not being considered an unsupervised classifier where accuracy
may be used~\cite{Huang1998}. With that, the first example is to use this
inertia as the fitness function in EDO.\ That is, to find datasets which
minimise \(I\).

For the purposes of visualisation, in this example EDO is restricted to only
two-dimensional datasets, i.e.\ \(C = \left((2, 2)\right)\). In addition to
this, all columns are formed from uniform distributions where the bounds are
sampled from the unit interval. Thus, the only family in \(\mathcal{P}\) is:
\begin{equation}
    \mathcal{U} := \left\{U(a, b)~|~a, b \in [0, 1]\right\}
\end{equation}

The remaining parameters are as follows: \(N~=~100\), \(R~=~(3, 100)\),
\(M~=~1000\), \(b~=~0.2\), \(l~=~0\), \(p_m~=~0.01\), and shrinkage excluded.
Figure~\ref{fig:small-inertia-50} shows an example of the fitness (above) and
dimension (below) progression of the evolutionary algorithm under these
conditions up until the \(50^{th}\) epoch.

There is a steep learning curve here; within the first 50 generations an
individual is found with a fitness of roughly \(10^{-10}\) which could not be
improved on for a further 900 epochs. The same quick convergence is seen in the
number of rows. This behaviour is quickly recognised as preferable and was
dominant across all the trials conducted in this work. This preference for
datasets with fewer rows makes sense given that \(I\) is the sum of the mean
error from each cluster centre. With that, when \(k\) is
fixed \textit{a priori}, reducing the number of points in each cluster (i.e.\
the terms of the second summation) quickly reduces the mean error of that
cluster and thus the value of \(I\). 

\addtocounter{figure}{1}
\begin{figure}[htbp]
    \ContinuedFloat%
    \centering
    \begin{tabular}{c}
        \includegraphics[width=\imgwidth]{small-inertia-fitness-50.pdf}
        \\
        \includegraphics[width=\imgwidth]{small-inertia-nrows-50.pdf}
    \end{tabular}
    \caption{%
        Progressions for final inertia and dimension across the first 50
        epochs with \(R~=~(3,100)\).
    }\label{fig:small-inertia-50}
\end{figure}

\begin{figure}[htbp]
    \ContinuedFloat%
    \centering
    \begin{tabular}{c}
        \includegraphics[width=\imgwidth]{large-inertia-fitness-50.pdf}
        \\
        \includegraphics[width=\imgwidth]{large-inertia-nrows-50.pdf}
    \end{tabular}
    \caption{%
        Progressions for final inertia and dimension across the first 50 epochs
        with \(R~=~(50,100)\).
    }\label{fig:large-inertia-50}
\end{figure}

Something that may be seen as unwanted is a compaction of the cluster centres.
Referring to Figure~\ref{fig:small-inertia-inds}, the best and median
individuals show two clusters that are essentially the same point whereas the
worst is a random cloud across the whole of \(\mathcal{U}\) which was found in
the initial population. The kind of behaviour exhibited by the best performing
individuals occurs in part because it is allowed. There are two immediate ways
in which this allowed: first, that the `trivial' case is included in \(R\)
and, secondly, that the fitness function does nothing to penalise the proximity
of the inter-cluster means, as well as aiming to reduce the intra-cluster means.
This kind of unwanted behaviour highlights a subtlety in how EDO should be used;
that experimentation and rigour are required to properly understand an
algorithm's quality.

\begin{figure}[htbp]
    \centering
    \subfloat[][]{%
        \label{fig:small-inertia-inds}
        \centering
        \includegraphics[width=\imgwidth]{small-inertia-individuals.pdf}
    }\\

    \subfloat[][]{%
        \label{fig:large-inertia-inds}
        \centering
        \includegraphics[width=\imgwidth]{large-inertia-individuals.pdf}
    }
    \caption[]{%
        Representative individuals based on inertia with:
        \subref{fig:small-inertia-inds} \(R~=~(3,100)\);
        \subref{fig:large-inertia-inds} \(R~=~(50,100)\). Centroids displayed as
        crosses.
    }\label{fig:inertia-inds}
\end{figure}

Hence, consider Figure~\ref{fig:large-inertia-inds} where the individuals have
been generated with the same parameters as previously except with adjusted row
limits, \(R = (50, 100)\), so as to exclude this trivial case. In these trials,
the results are equivalent: the worst performing individuals are without
structure whilst the best-performing individuals display clusters that are dense
about a single point despite the minimum number of rows being increased. Perhaps
then, this compacted clustering is `optimal'.

However, more extensive studying may be done. That is, the defined fitness
function may require further attention. Indeed, the final inertia could be
considered a flawed or fragile fitness function if it is supposed to evaluate
the appropriateness or efficacy of the \(k\)-means algorithm. Incorporating the
inter-cluster spread to the fitness of an individual dataset can reduce this
observed compaction. The silhouette coefficient is a metric used to evaluate the
appropriateness of a clustering to a dataset, and is given by the mean of the
silhouette value, \(S(x)\), of each point \(x \in Z_j\) in each cluster:
\begin{equation}
    \begin{gathered}
        A(x) := \frac{1}{|Z_j| - 1} \sum_{y \in Z_j \setminus \{x\}} d(x, y),
        \\
        B(x) := \min_{k \neq j} \frac{1}{|Z_k|} \sum_{w \in Z_k} d(x, w),
        \\
        S(x) := 
            \begin{cases}
                \frac{B(x) - A(x)}{\max\left\{A(x), B(x)\right\}}
                &\quad \text{if } |Z_j| > 1\\
                0 &\quad \text{otherwise}
            \end{cases}
    \end{gathered}\label{eq:silhouette}
\end{equation}\\

The optimisation of the silhouette coefficient is analogous to finding a dataset
which increases both the intra-cluster cohesion (the inverse of \(A\)) and
inter-cluster separation (\(B\)). Hence, the inertia is addressed by maximising
cohesion. Meanwhile, the spread of the clusters themselves is considered by
maximising separation.

Repeating the trials with the same parameters as with inertia, the silhouette
fitness function yields the results summarised in
Figures~\ref{fig:small-silhouette}~and~\ref{fig:large-silhouette}. Irrespective
of row limits, the datasets produced show increased separation from one another
whilst maintaining low values in the final inertia of the clustering as shown in
Figure~\ref{fig:silhouette-inds}. Again, the form of the individual clusters is
much the same. The low values of inertia correspond to tight clusters, and the
tightest clusters are those with a minimal number of points, i.e.\ a single
point. As with the previous example, albeit at a much slower rate, the
preferable individuals are those leading toward this case. That this gradual
reduction in the dimension of the individuals occurs after the improvement of
the fitness function bolsters the claim that the base case is also optimal.

However, due to the nature of the implementation, any individual from any
generation may be retrieved and studied should the final results be too
concentrated on any given case. This transparency in the history and progression
of the proposed method is something that sets it apart from other methods of the
same ilk such as GANs which have a reputation of providing so-called `black
box' solutions.

\addtocounter{figure}{1}
\begin{figure}[htbp]
    \ContinuedFloat%
    \centering
    \begin{tabular}{c}
        \includegraphics[width=\imgwidth]{small-silhouette-fitness.pdf}
        \\
        \includegraphics[width=\imgwidth]{small-silhouette-nrows.pdf}
    \end{tabular}
    \caption{%
        Progressions for silhouette and dimension across 1000 epochs at 100
        epoch intervals with \(R~=~(3, 100)\).
    }\label{fig:small-silhouette}
\end{figure}

\begin{figure}
    \ContinuedFloat%
    \centering
    \begin{tabular}{c}
        \includegraphics[width=\imgwidth]{large-silhouette-fitness.pdf}
        \\
        \includegraphics[width=\imgwidth]{large-silhouette-nrows.pdf}
    \end{tabular}
    \caption{%
        Progressions for silhouette and dimension across 1000 epochs at 100
        epoch intervals with \(R~=~(50,100)\).
    }\label{fig:large-silhouette}
\end{figure}

\begin{figure}[htbp]
    \centering
    \subfloat[][]{%
        \label{fig:small-silhouette-inds}
        \centering
        \includegraphics[width=\imgwidth]{small-silhouette-individuals.pdf}
    }\\

    \subfloat[][]{%
        \label{fig:large-silhouette-inds}
        \centering
        \includegraphics[width=\imgwidth]{large-silhouette-individuals.pdf}
    }
    \caption[]{%
        Representative individuals based on silhouette with:
        \subref{fig:small-silhouette-inds} \(R~=~(3,100)\);
        \subref{fig:large-silhouette-inds} \(R~=~(50,100)\). Centroids displayed
        as crosses.
    }\label{fig:silhouette-inds}
\end{figure}


\subsection{Comparison with DBSCAN}

The capabilities of EDO as a tool for understanding an algorithm are highlighted
particularly when comparing an algorithm against another (or set of others)
simultaneously. This is done by utilising the freedom of choice in a fitness
function for EDO.\ Consider two algorithms, \(A\) and \(B\), and some common
metric between them, \(g\). Then understanding their similarities and contrasts
can be done by considering the differences in this metric on the two algorithms.
In terms of EDO, this means using \(f = g_A - g_B\), \(f = g_B - g_A\) or \(f
= \left| g_B - g_A \right|\) as the fitness function. By doing so, pitfalls,
edge cases or fundamental conditions for the method can be highlighted.
Overall, this process allows the researcher to more deeply learn about the
method of interest.

As an example of this process, consider the another clustering algorithm of a
different form such as Density Based Spatial Clustering of Applications with
Noise (DBSCAN) and suppose the objective is to find datasets for which
\(k\)-means outperforms this alternative. Here there is no concept of inertia as
DBSCAN is density-based and is able to identify outliers~\cite{Ester1996}. As
such, a valid must be chosen. One such metric is the silhouette score as
defined in~(\ref{eq:silhouette}).

However, an adjustment to the fitness function must be made so as to accommodate
for the condition of the silhouette coefficient that there be more than one
cluster present. Let \(S_k (X)\) and \(S_D (X)\) denote the silhouette
coefficients of the clustering found by \(k\)-means and DBSCAN respectively.
Then the fitness function is defined to be:
\begin{equation}
    f(X) = 
        \begin{cases}
            S_D (X) - S_k (X), &\quad \text{%
                \begin{tabular}{l}
                    if DBSCAN identifies two or
                    \\
                    more clusters (inc.\ noise)
                \end{tabular}
            }\\
            \infty &\quad \text{otherwise.}
        \end{cases}\label{eq:dbscan-fitness}
\end{equation}

There are several remarks to be made here. First, note the order of the
subtraction here as EDO minimises fitness functions by default. Also, \(f\)
takes values in the range \([-2, 2]\) where \(-2\) is the best, i.e.\ \(S_D(X) =
-1\) and \(S_k(X) = 1\). Likewise, 2 is the worst score. Finally, the silhouette
coefficient requires at least two clusters to be present and so if DBSCAN
identifies a single cluster then that individual will be penalised heavily under
this fitness function when, in fact, that clustering may be of high quality. As
such, this fitness function may require adjustment.

It must also be acknowledged that \(k\)-means and DBSCAN share no common
parameters and so direct comparison is more difficult. For the purposes of this
example, only one set of parameters is used but a thorough investigation should
include a parameter sweep in cases such as these. The parameters being used are
\(k~=~3\) for \(k\)-means, and \(\epsilon~=~0.1,\ MinPoints~=~5\) for DBSCAN.\
This set was chosen following informal experimentation using the Python library
Scikit-learn~\cite{scikit} to find comparable parameters in the given search
space defined by the EDO parameters used previously with \(R~=~(50,100)\).

\begin{figure}[htbp]
    \centering
    \begin{minipage}{\imgwidth}
        \centering
        \includegraphics[width=\imgwidth]{dbscan-fitness.pdf}
    \end{minipage}

    \begin{minipage}{\imgwidth}
        \centering
        \includegraphics[width=\imgwidth]{dbscan-nrows.pdf}
    \end{minipage}
    \caption{%
        Progressions for difference in silhouette (\(k\)-means-preferable) and
        dimension across 1000 epochs at 100 epoch intervals.
    }\label{fig:dbscan-silhouette}
\end{figure}

\begin{figure}[htbp]
    \centering
    \subfloat[][]{%
        \label{fig:dbscan-inds-k}
        \centering
        \includegraphics[width=\imgwidth]{kmeans-individuals.pdf}
    }\\
    \subfloat[][]{%
        \label{fig:dbscan-inds-d}
        \centering
        \includegraphics[width=\imgwidth]{dbscan-individuals.pdf}
    }
    \caption[]{%
        Representative individuals from a \(k\)-means-preferable run with
        clustering by: \subref{fig:dbscan-inds-k} \(k\)-means;
        \subref{fig:dbscan-inds-d} DBSCAN.\ Concave and convex hulls illustrated
        by shading and outline respectively. 
    }\label{fig:dbscan-inds}
\end{figure}

Figure~\ref{fig:dbscan-silhouette} shows a summary of the progression of EDO
for this use case. As with the previous examples where \(R~=~(50, 100)\), the
variation in the population fitness is unstable but there is a clear trend of
improvement in the best individual over the course of the run. There is also a
convergence seen in the number of rows a dataset has. The resting dimension
varied across the trials conducted in this work but none exhibited a shift
toward the lower limit of 50 rows as with previous examples. This is suggestive
of a more competitive environment for individuals where slight changes to an
individual can drastically alter their fitness.

The effect of such changes can be seen in Figure~\ref{fig:dbscan-inds} where
representative individuals are shown for this example. Here, the best performing
individual, when clustered by \(k\)-means, shows three clear and nicely
separated clusters. Note that they are not so tightly packed; again, this
suggests that the route to an optimal individual is less clearly defined. In
contrast, when the same dataset is clustered by DBSCAN a single cluster is found
with a single noise point held within the convex hull of the cluster, i.e.\
there are overlapping clusters (since noise points form a single cluster).
Hence, along with the fact that the larger cluster is widely spread, it follows
that the clustering have a relatively small, negative silhouette coefficient.

Another point of interest here is the convexity of the clusters. One of the
conditions for the success of \(k\)-means is that the presented clusters are of
roughly equal size and are convex. This is due to the overall objective being to
approximate the centroidal Voronoi tessellation~\cite{Du2006}. Without this
condition, up to the correct choice of \(k\), the algorithm will fail to produce
adequate results for either inertia or silhouette. DBSCAN, however, does not
have this condition and is able to detect non-convex clusters so long as they
are dense enough. Figure~\ref{fig:dbscan-inds} shows the both the clustering and
the convex and concave hulls of the clusters found by each method. The `concave
hull' of a cluster is taken to be the \(\alpha\)-shape of the cluster's data
points~\cite{Edelsbrunner1983} where \(\alpha\) is determined to be the smallest
value such that all the points in the cluster are contained in a single polygon.
The convexity of cluster \(Z_j\), denoted \(\mathcal{C}_j\), is then determined
to be the ratio of the area of its concave hull, \(H_c\), to the area of its
convex hull, \(H_v\)~\cite{Sonka1993}:
\begin{equation}
    \mathcal{C}_j := \frac{area(H_c)}{area(H_v)}
\end{equation}

With this definition, it should be clear that a perfectly convex cluster, such
as a single point or line, would have \(\mathcal{C}_j = 1\).

It can be seen that the convexity of the clustering found by \(k\)-means appears
to be higher than that by DBSCAN.\ This was apparent across all trials conducted
in this work and indicates that the condition for convex clusters is being
sought out during the optimisation process. Meanwhile, however, it is not clear
whether the performance of DBSCAN falls owing to its parameters or the method
itself. This is a point where parameter sweeping would prove most useful so as
to determine a crossing point for these two driving forces.

% TODO Perhaps some statistical test could be used here, or with the second part
% of this example. Would require lots more trials (approx one week to run and
% summarise the data)

Now, to add to the discussion above, the inverse optimisation should be
considered. That is, using the same parameters, the datasets for which DBSCAN
outperforms \(k\)-means with respect to the silhouette coefficient are to be
investigated. This is equivalent to using \(-f\) as the fitness function
except with the same penalty of \(\infty\) for the case set out
in~(\ref{eq:dbscan-fitness}).

\begin{figure}[htbp]
    \centering
    \begin{tabular}{c}
        \includegraphics[width=\imgwidth]{negative-fitness.pdf}
        \\
        \includegraphics[width=\imgwidth]{negative-nrows.pdf}
    \end{tabular}
    \caption{%
        Progressions for difference in silhouette (DBSCAN-preferable) and
        dimension across 1000 epochs at 100 epoch intervals.
    }\label{fig:negative-prog}
\end{figure}

\begin{figure}[htbp]
    \centering
    \subfloat[][]{%
        \label{fig:neg-inds-k}
        \centering
        \includegraphics[width=\imgwidth]{negative-kmeans-individuals.pdf}
    }\\
    \subfloat[][]{%
        \label{fig:neg-inds-d}
        \centering
        \includegraphics[width=\imgwidth]{negative-dbscan-individuals.pdf}
    }
    \caption[]{%
        Representative individuals from a DBSCAN-preferable run with clustering
        by: \subref{fig:dbscan-inds-k} \(k\)-means; \subref{fig:dbscan-inds-d}
        DBSCAN.\ Concave and convex hulls illustrated by shading and outline
        respectively. 
    }\label{fig:negative-inds}
\end{figure}

Figures~\ref{fig:negative-prog}~and~\ref{fig:negative-inds} show the same
summary as above with the revised fitness function. Inspecting the former, it is
seen that the best fitness found is worse than with the previous example. This,
in part, is due to the fact that \(k\)-means cannot find a clustering with
negative values as no clusters may overlap. The method can, however, produce
results with small silhouette scores where the clusters are tightly packed.
Hence, the best fitness score is now \(-1\) whereas the worst is 2, still.

Note in the first two frames of Figure~\ref{fig:neg-inds-k} how \(k\)-means is
forced to split what is evidently a single cluster in two whereas DBSCAN is able
to identify the single cluster and the outlying noise
(Figure~\ref{fig:neg-inds-d}). The proximity of these clusters has then dragged
the silhouette score down for \(k\)-means. Referring to
Figure~\ref{fig:neg-inds-d}, this kind of behaviour is certainly preferable for
DBSCAN under these parameters: the beginning individuals are likely random
clouds (as seen in the rightmost two frames of the figure) and the simplest step
toward a fit dataset is one that maintains that vaguely dense body with minimal
noise points far from it.
