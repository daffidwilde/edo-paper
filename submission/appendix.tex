\appendix
\section{Appendix}

\subsection{Lloyd's algorithm}\label{app:kmeans}

\input{alg-kmeans.tex}

\subsection{Implementation example}\label{app:code}

Below is an example of how the Python implementation was used to complete the
first example, including the definition of the fitness function.

\begin{lstlisting}
import edo
from edo.pdfs import Uniform
from sklearn.cluster import KMeans


def fitness(dataframe, seed):
    """ Return the final inertia of 2-means on the `dataframe`
    for the given `seed`. """

    km = KMeans(n_clusters=2, random_state=seed).fit(dataframe)
    return km.inertia_


Uniform.param_limits["bounds"] = [0, 1]

pop_history, fit_history = edo.run_algorithm(
    fitness, size=100, row_limits=[3, 100], col_limits=[2, 2],
    families=[Uniform], max_iter=1000, best_prop=0.2,
    mutation_prob=0.01, seed=0, root="out",
    fitness_kwargs={"seed": 0},
)
\end{lstlisting}

