\section{Conclusion}

In this paper we have introduced a novel approach to understanding the quality
of an algorithm by exploring the space in which their well-performing datasets
exist. Following a detailed explanation of its internal mechanisms, a case study
in \(k\)-means clustering was offered as validation for the method. The method
utilises biological operators to traverse the space of all possible datasets in
an organic way with a minimal external framework attached. The generative nature
of the proposed method also provides transparency and richness to the solution
when compared to other contemporary techniques for artificial data generation as
the entire history of individuals is preserved.

The evolutionary dataset optimisation method is dependent on a number of
parameters set out in this paper and perhaps the most important of which is the
choice of distribution families, \(\mathcal{P}\); these families set out the
general statistical shape of the columns of the datasets that are produced and
also control the present data types. The relationship between columns and their
associated distribution is not causal and appropriate methods should be
employed to understand the structure and characteristics of the data produced
before formal conclusions are made as set out in the examples provided.
