\documentclass{article}

\usepackage{fullpage}

\usepackage[ruled]{algorithm2e}
\usepackage{amssymb}
\usepackage{amsmath}
\usepackage{amsthm}
\usepackage[backend=bibtex, style=numeric, giveninits=true]{biblatex}
    \addbibresource{references}
\usepackage{booktabs} % Pandas dataframes
\usepackage{caption}
\usepackage{enumerate} % Roman numerals
\usepackage{float}
\usepackage{graphicx}
\usepackage{hyperref}
\usepackage{mathptmx}
\usepackage{minted}
\usepackage{standalone}
\usepackage{subfig}
\usepackage{tikz}
    \usetikzlibrary{%
        arrows.meta,
        decorations.pathreplacing,
        decorations.text,
        patterns,
        shapes.arrows,
        shapes.geometric
    }


% Page setup and lengths
\raggedbottom%

\newlength{\imgwidth}
\setlength{\imgwidth}{.95\textwidth}

\definecolor{cyan}{RGB}{0, 164, 216}
\definecolor{magenta}{RGB}{226, 62, 138}


% Algorithms
\newcommand{\balg}[1][htbp]{%
    \begin{algorithm}[#1]\DontPrintSemicolon
}
\newcommand{\ealg}{\end{algorithm}}


% TikZ styles, commands and settings
\pgfdeclarelayer{background}
\pgfsetlayers{background,main}

\tikzstyle{every picture} += [remember picture]
\tikzstyle{na} = [baseline=-.5ex]

\tikzset{%
    column/.pic={%
        code{%
            \draw[line width=1pt] (0, 0) rectangle (-2cm, 4cm);
            \foreach \val in {0, ..., #1}{%
                \draw[rotate=90] ([xshift=-\val*10pt] 4cm, 2cm) -- ++(0, -2cm);
            };
            \node at (-1cm, 1.25) {$\vdots$};
            \foreach \val in {1, 2}{%
                \draw (0, \val * 10pt) -- ++(-2cm, 0);
            };
        }
    }
}

\tikzset{%
    fullcolumn/.pic={%
        code{%
            \draw[line width=1pt] (0, 0) rectangle (-2cm, #1*10pt);
            \foreach \val in {0, ..., #1}{%
                \draw[rotate=90] ([xshift=-\val*10pt] #1*10pt, 2cm) -- ++(0, -2cm);
            };
        }
    }
}

\newcommand{\inputtikz}[3][.8\linewidth]{%
    \begin{figure}[htbp]
        \centering
        \resizebox{#1}{!}{%
            \input{tex/diagrams/#2.tex}
        }
        \caption{#3}\label{fig:#2}
    \end{figure}
}

\DeclareMathOperator*{\argmin}{arg\,min}
\renewcommand\theContinuedFloat{\alph{ContinuedFloat}}

% Document
\title{%
    Evolutionary Dataset Optimisation:
    learning algorithm quality through evolution
}
\author{Henry Wilde, Vincent Knight, Jonathan Gillard}


\begin{document}
\maketitle%

\begin{abstract}
    In this paper we propose a new method for learning how algorithms perform.
    Classically, algorithms are compared on a finite number of existing (or
    newly simulated) benchmark data sets based on some fixed metric. The
    algorithm(s) with the smallest value of this metric are chosen to be the
    `best performing'.

    We offer a new approach to flip this paradigm. We instead aim to gain a
    richer picture of the performance of an algorithm by generating artificial
    data through genetic evolution, the purpose of which is to create
    populations of datasets for which a particular algorithm performs well.
    These data sets can be studied to learn as to what attributes lead to a
    particular progress of a given algorithm.

    Following a detailed description of the algorithm as well as a brief
    description of an open source implementation, a number of numeric
    experiments are presented to show the performance of the method.
\end{abstract}


\graphicspath{{./img/}}
\section{Introduction}\label{section:introduction}

\inputtikz{flowchart.tex}{%
    A general schematic for an evolutionary algorithm.
}

\begin{itemize}
    \item What is the motivation?
    \item What is the problem?
    \item What is the solution?
\end{itemize}

%-----------------------------
\subsection{Literature review}

\begin{itemize}
    \item How is artificial data made?
    \item Why hasn't this been done before?
    \item Genetic algorithms used to train algorithms for data
    \item Diagram showing that this is the ``reverse'' problem
\end{itemize}

\section{The evolutionary algorithm}\label{section:algorithm}

\subsection{Structure}

In this section, the details of an algorithm that generates data for which a
given function or, equivalently, an algorithm which is well suited, is
described. This algorithm is to be referred to as ``Evolutionary Dataset
Optimisation'' (EDO).

The EDO method is built as an evolutionary algorithm which follows a traditional
(generic) schema with some additional features that keep the objective of
artificial data generation in mind. With that, there are a number of parameters
that are passed to EDO;\ the typical parameters of an evolutionary algorithm
are a fitness function, \(f\), which maps from an individual to a real number,
as well as a population size, \(N\), a maximum number of iterations, \(M\), a
selection parameter, \(b\), and a  mutation probability, \(p_m\). In addition to
these, EDO takes the following parameters:

\begin{itemize}
    \item Limits on the number of rows an individual dataset can have:
        \[
            R \in \left\{%
                (r_{\min}, r_{\max}) \in \mathbb{N}^2~|~r_{\min} \leq r_{\max}
            \right\}
        \]
    \item Limits on the number of columns a dataset can have:
        \[
            C := \left(C_1, \ldots, C_{|\mathcal{P}|}\right)
            ~\text{where}~
            C_j \in \left\{ (c_{\min}, c_{\max}) \in {%
                \left(\mathbb{N}\cup\{\infty\}\right)
            }^2~|~c_{\min} \leq c_{\max}\right\}
        \]
        for each \(j = 1, \ldots, |\mathcal{P}|\). That is, \(C\) defines the
        minimum and maximum number of columns a dataset may have from each
        distribution in \(\mathcal{P}\).
    \item A set of probability distribution families, \(\mathcal{P}\). Each
        family in this set has some parameter limits which form a part of the
        overall search space. For instance, the family of normal distributions,
        denoted by \(N(\mu, \sigma^2)\), would have limits on values for the
        mean, \(\mu\), and the standard deviation, \(\sigma\).
    \item A probability vector to sample distributions from \(\mathcal{P}\),
        \(w = \left(w_1, \ldots, w_{|\mathcal{P}|}\right)\).
    \item A second selection parameter, \(l \in [0, 1]\), to allow for a
        small proportion of ``lucky'' individuals to be carried forward.
    \item A shrink factor, \(s \in [0, 1]\), defining the relative size of a
        component of the search space to be retained after adjustment.
\end{itemize}

The concepts discussed in this section form the mechanisms of the evolutionary
dataset optimisation algorithm. To use the algorithm practically, these
components have been implemented in Python as a library built on the scientific
Python stack~\cite{pandas,numpy}. The library is fully tested and documented (at
\url{https://edo.readthedocs.io}) and is freely available online under the MIT
license~\cite{edo-project}. The EDO implementation was developed to be
consistent with the best practices of open source software development.
% TODO Citation(s) needed from Vince.

\balg%
\KwData{\(f, N, R, C, \mathcal{P}, w, M, b, l, \mu, s\)}
\KwResult{A full history of the populations and their fitnesses.}

\Begin{%
    create initial population of individuals\;
    find fitness of each individual\;
    record population and its fitness\;

    \While{%
        current iteration less than the maximum
        \textbf{and} stopping condition not met
    }{%
        select parents based on fitness and selection proportions\;
        use parents to create new population through crossover and mutation\;
        find fitness of each individual\;
        update population and fitness histories\;
        \If{adjusting the mutation probability}{%
            update mutation probability
        }
        \If{using a shrink factor}{%
            shrink the mutation space based on parents
        }
    }
}
\caption{The evolutionary dataset optimisation algorithm}
\ealg%
\label{alg:edo}
\balg%
\KwData{parents, \(N, R, C, \mathcal{P}, w, p_m\)}
\KwResult{A new population of size \(N\)}

\Begin{%
    add parents to the new population\;
    \While{the size of the new population is less than \(N\)}{%
        sample two parents at random\;
        create an offspring by crossing over the two parents\;
        mutate the offspring according to the mutation probability\;
        add the mutated offspring to the population\;
    }
}
\caption{Creating a new population}
\ealg%


The statement of the EDO algorithm is presented here to lay out its general
structure from a high level perspective. Lower level discussion is provided
below where additional algorithms for the individual creation, evolutionary
operator and shrinkage processes are given along with diagrams (where
appropriate).

Note that there are no defined processes for how to stop the algorithm or adjust
the mutation probability, \(p_m\). This is down to their relevance to a
particular use case. Some examples include:

\begin{itemize}
    \item Stopping when no improvement in the best fitness is found within some
        \(K\) consecutive iterations~\cite{Leung2001}.
    \item Utilising global behaviours in fitness to indicate a stopping
        point~\cite{Marti2016}.
    \item Regular decreasing in mutation probability across the available
        attributes~\cite{Kuehn2013}.
\end{itemize}


\subsection{Individuals}

Evolutionary algorithms operate in an iterative process on populations of
individuals that each represent a solution to the problem in question. In a
genetic algorithm, an individual is a solution encoded as a bit string of,
typically, fixed length and treated as a chromosome-like object to be
manipulated. In EDO, as the objective is to generate datasets and explore the
space in which datasets exist, there is no encoding. As such the distinction is
made that EDO is an evolutionary algorithm. 

As is seen in Figure~\ref{figure:individual}, an individual's creation is
defined by the generation of its columns. A set of instructions on how to sample
new values (in mutation, for instance, Section~\ref{subsection:mutation}) for
that column are recorded in the form of a probability distribution. These
distributions are sampled and created from the families passed in
\(\mathcal{P}\). In EDO, the produced datasets and their metadata are
manipulated directly so that the biological operators can be designed and be
interpreted in a more meaningful way as will be seen later in this section.

However, one should not assume that the columns are a reliable representative of
the distribution associated with them, or vice versa. This is particularly true
of ``shorter'' datasets with a small number of rows, whereas confidence in the
pair could be given more liberally for ``longer'' datasets with a larger number
of rows. In any case, appropriate methods for analysis should be employed before
formal conclusions are made.

\inputtikz{individual}{%
    An example of how an individual is first created.
}
\balg%
\KwData{\(R, C, \mathcal{P}, w\)}
\KwResult{An individual defined by a dataset and some metadata}

\Begin{%
    sample a number of rows and columns\;
    create an empty dataset\;
    \For{each column in the dataset}{%
        sample a distribution from \(\mathcal{P}\)\;
        create an instance of the distribution\;
        fill in the column by sampling from this instance\;
        record the instance in the metadata
    }
}
\caption{Creating an individual}
\ealg%



\subsection{Selection}

The selection operator describes the process by which individuals are chosen
from the current population to generate the next. Almost always, the likelihood
of an individual being selected is determined by their fitness. This is because
the purpose of selection is to preserve favourable qualities and encourage some
homogeneity within future generations~\cite{Back1994}.

\inputtikz[.8\imgwidth]{selection}{%
    The selection process with the inclusion of some lucky individuals.
}
\balg%
\KwIn{population, population fitness, \(b\), \(l\)}
\KwOut{A set of parent individuals}

\Begin{%
    calculate \(n_b\) and \(n_l\)\;
    sort the population by the fitness of its individuals\;
    take the first \(n_b\) individuals and make them parents\;
    \If{there are any individuals left}{%
        take the next \(n_l\) individuals and make them parents\;
    }
}
\caption{The selection process}\label{alg:selection}
\ealg%


In EDO, a modified truncation selection method is used~\cite{Jebari2013}, as can
be seen in Figure~\ref{figure:selection}. Truncation selection takes a fixed
number, \(n_b = \lceil bN\rceil\), of the fittest individuals in a population
and makes them the ``parents'' of the next. It has been observed that, despite
its efficiency as a selection operator, truncation selection can lead to
premature convergence at local optima~\cite{Jebari2013, Tatsuya2002}. The
modification for EDO is an optional stage after the best individuals have been
chosen: with some small \(l\), a number, \(n_l = \lceil lN\rceil\), of the
remaining individuals can be selected at random to be carried forward. Hence,
allowing for a small number of randomly selected individuals may encourage
diversity and further exploration throughout the run of the algorithm. It should
be noted that regardless of this step, an individual could potentially be
present throughout the entirety of the algorithm.

After the parents have been selected, there are two adjustments made to the
current search space. The first is that the subtypes for each family in
\(\mathcal{P}\) are updated to only those present in the parents. The second
adjustment is a process which acts on the distribution parameter limits for
each subtype in \(\mathcal{P}\). This adjustment gives the ability to ``shrink''
the search space about the region observed in a given population. This method is
based on a power law described in~\cite{Amirjanov2016} that relies on a shrink
factor, \(s\). At each iteration, \(t\), every distribution subtype which is
present in the parents has its parameter's limits, \(\left(l_t, u_t\right)\),
adjusted. This adjustment is such that the new limits, \(\left(l_{t+1},
u_{t+1}\right)\) are centred about the mean observed value, \(\mu\), for that
parameter:
\begin{align}
    \label{eq:shrinking_lower}
    l_{t+1}&= \max \left\{l_t, \ \mu - \frac{1}{2} (u_t - l_t) s^t\right\}\\
    \label{eq:shrinking_upper}
    u_{t+1}&= \min \left\{u_t, \ \mu + \frac{1}{2} (u_t - l_t) s^t\right\}
\end{align}

The shrinking process is given explicitly in
Algorithm~\ref{algorithm:shrinking}. Note that the behaviour of this process can 
produce reductive results for some use cases and is optional.

\balg%
\KwIn{parents, current iteration, \(\mathcal{P}, M, s\)}
\KwOut{A new mutation space focussed around the parents}

\Begin{%
    \For{each distribution subtype in \(\mathcal{P}\)}{%
        \For{each parameter of the distribution}{%
            get the current values for parameter over all parent columns\;
            find the mean of the current values\;
            find the new lower~(\ref{eq:shrinking_lower}) and
            upper~(\ref{eq:shrinking_upper}) bounds around the mean\;
            set the parameter limits\;
        }
    }
}
\caption{Shrinking the mutation space}\label{alg:shrinking}
\ealg%



\subsection{Crossover}

Crossover is the operation of combining two individuals in order to create at
least one offspring. In genetic algorithms, the term ``crossover'' can be taken
literally: two bit strings are crossed at a point to create two new bit strings.
Another popular method is uniform crossover, which has been favoured for its
efficiency and efficacy in combining individuals~\cite{Semenkin2012}. For EDO,
this method is adapted to support dataset manipulation: a new individual is
created by uniformly sampling each of its components (dimensions and then
columns) from a set of two ``parent'' individuals, as shown in
Figure~\ref{figure:crossover}.

\inputtikz{crossover}{%
    The crossover process between two individuals with different dimensions.
}

Observe that there is no requirement on the dimensions of the parents to be of
similar or equal shapes. This is because the driving aim of the proposed method
is to explore the space of all possible datasets. In the case where there is
incongruence in the lengths of the two parents, missing values may appear in a
shorter column that is sampled. To resolve this, values are sampled from the
probability distribution associated with that column to fill in these gaps.

\balg%
\KwIn{Two parents}
\KwOut{An offspring made from the parents ready for mutation}

\Begin{%
    collate the columns and metadata from each parent in a pool\;
    sample each dimension from between the parents uniformly\;
    form an empty dataset with these dimensions\;
    \For{each column in the dataset}{%
        sample a column (and its corresponding metadata) from the pool\;
        \If{this column is longer than required}{%
            randomly select entries and delete them as needed 
        }
        \If{this column is shorter than required}{%
            sample new values from the metadata and append them to the column as
            needed
        }
        add this column to the dataset and record its metadata\;
    }
}
\caption{The crossover process}
\ealg%



\subsection{Mutation}

Mutation is used in evolutionary algorithms to encourage a broader exploration
of the search space at each generation. Under this framework, the mutation
process manipulates the phenotype of an individual where numerous things need to
be modified including an individual's dimensions, column metadata and the
entries themselves. This process is described in Figure~\ref{figure:mutation}.

\inputtikz{mutation}{The mutation process.}

As shown in Figure~\ref{figure:mutation}, each of the potential mutations occur
with the same probability, \(p_m\). However, the way in which columns are
maintained assure that (assuming appropriate choices for \(f\) and
\(\mathcal{P}\)) many mutations in the metadata and the dataset itself will only
result in some incremental change in the individual's fitness
relative to, say, a completely new individual.

\balg%
\KwData{An individual, \(p_m\), \(R\), \(C\), \(\mathcal{P}\), \(w\)}
\KwResult{A mutated individual}

\Begin{%
    sample a random number \(r \in [0, 1]\)\;
    \If{\(r < p_m\) and adding a row would not violate \(R\)}{%
        sample a value from each distribution in the metadata\;
        append these values as a row to the end of the dataset\;
    }
    sample a new \(r \in [0, 1]\)\;
    \If{\(r < p_m\) and removing a row would not violate \(R\)}{%
        remove a row at random from the dataset
    }
    sample a new \(r \in [0, 1]\)\;
    \If{\(r < p_m\) and adding a new column would not violate \(C\)}{%
        create a new column using \(\mathcal{P}\) and \(w\)\;
        append this column to the end of the dataset
    }
    sample a new \(r \in [0, 1]\)\;
    \If{\(r < p_m\) and removing a column would not violate \(C\)}{%
        remove a column (and its associated metadata) at random from the dataset
    }
    \For{each distribution in the metadata}{%
        \For{each parameter of the distribution}{%
            sample a random number \(r \in [0, 1]\)\;
            \If{\(r < p_m\)}{%
                sample a new value from within the distribution parameter
                limits\;
                update the parameter value with this new value
            }
        }
    }
    \For{each entry in the dataset}{%
        sample a random number \(r \in [0, 1]\)\;
        \If{\(r < p_m\)}{%
            sample a new value from the associated column distribution\;
            update the entry with this new value
        }
    }
}
\caption{The mutation process}\label{algorithm:mutation}
\ealg%



\section{Examples}\label{section:examples}

\subsection{\(k\)-means clustering}

The following examples act as a form of validation for EDO, and also highlight
some of the nuances in its use. The examples will be focused around the
clustering of data and, in particular, the \(k\)-means (Lloyd's) algorithm.
Clustering was chosen as it is a well-understood problem that is easily
accessible \-- especially when restricted to two dimensions. The \(k\)-means
algorithm is an iterative, centroid-based method that aims to minimise the
``inertia'' of the current partition, \(Z = \left\{Z_1, \ldots, Z_k\right\}\),
of some dataset \(X\):

\begin{equation}
    I(Z, X) := \frac{1}{|X|} \sum_{j=1}^{k} \sum_{x \in Z_j} {d(x, z_j)}^2
\end{equation}

A full statement of the algorithm is given in~\ref{appendix:kmeans}. However, it
should be clear that \(I\) may take any non-negative value.

This inertia function is often taken as the objective of the \(k\)-means
algorithm, and is used for evaluating the final clustering. This is particularly
true when the algorithm is not being considered an unsupervised classifier where
accuracy may be used~\cite{Huang1998}. With that, the first example is to use
this inertia as the fitness function in EDO.\ That is, to find datasets which
minimise \(I\).

In this example, EDO is restricted to only two-dimensional datasets, i.e.\ \(C =
\left((2, 2)\right)\). In addition to this, all columns are formed from the
uniform distribution restricted to the unit interval, \(\mathcal{U} :=
\left\{U(a, b)~|~a, b \in [0, 1]\right\}\). The remaining parameters are as
follows: \(N~=~100\), \(R~=~(3, 100)\), \(M~=~1000\), \(b~=~0.2\), \(l~=~0\),
\(p_m~=~0.01\), and shrinkage excluded. Figure~\ref{figure:inertia-50} shows an
example of the fitness (above) and dimension (below) progression of the
evolutionary algorithm under these conditions up until the \(50^{th}\) epoch.

There is a steep learning curve here; within the first 50 generations an
individual is found with a fitness of roughly \(10^{-10}\) which could not be
improved on for a further 900 epochs. The same quick convergence is seen in the
number of rows. This behaviour is quickly recognised as preferable and was
dominant across all the trials conducted in this work. This preference for
datasets with fewer rows makes sense given that \(I\) is a summation of
non-negative terms (since \(d\) is a distance metric.) With that, when \(k\) is
fixed \textit{a priori}, reducing the number of terms in the second summation
quickly reduces the value of \(I\). 

\begin{figure}[htbp]
    \ContinuedFloat*
    \centering
    \begin{tabular}{c}
        \includegraphics[width=\imgwidth]{img/inertia-fitness-50.pdf}\\
        \includegraphics[width=\imgwidth]{img/inertia-nrows-50.pdf}
    \end{tabular}
    \caption{%
        \label{figure:inertia-50}
        Progressions for final inertia and dimension across the first 50
        epochs with \(R~=~(3,100)\).
    }
\end{figure}

\begin{figure}[htbp]
    \ContinuedFloat%
    \centering
    \begin{tabular}{c}
        \includegraphics[width=\imgwidth]{img/large-inertia-fitness-50.pdf}\\
        \includegraphics[width=\imgwidth]{img/large-inertia-nrows-50.pdf}
    \end{tabular}
    \caption{%
        \label{figure:large-inertia-50}
        Progressions for final inertia and dimension across the first 50 epochs
        with \(R~=~(50,100)\).
    }
\end{figure}

Something that may be seen as unwanted is a compaction of the clusters.
Referring to Figure~\ref{figure:inertia-individuals}, the best individual shows
two clusters but they are all essentially the same point whereas the worst is a
random cloud across the whole of \(\mathcal{U}\) which was found in the
initial population. The kind of behaviour exhibited by the best performing
individuals occurs in part because it is allowed. There are two immediate ways
in which this allowed: first, that the ``trivial'' case is included in \(R\)
and, secondly, that the fitness function does nothing to penalise the reduction
in the inter-cluster means, as well as the intra-cluster means. This kind of
unwanted behaviour highlights a subtlety in how EDO should be used;
experimentation and rigour are required to properly understand an algorithm's
quality.

\begin{figure}[htbp]
    \ContinuedFloat*
    \centering
    \includegraphics[width=\imgwidth]{img/inertia-individuals-0.pdf}
    \caption{%
        \label{figure:inertia-individuals}
        Representative individuals based on inertia with \(R~=~(3,100)\).
        Centroids displayed as crosses.
    }
\end{figure}

\begin{figure}[htbp]
    \ContinuedFloat%
    \centering
    \includegraphics[width=\imgwidth]{img/large-inertia-individuals-0.pdf}
    \caption{%
        \label{figure:large-inertia-individuals}
        Representative individuals based on inertia with \(R~=~(50,100)\).
        Centroids displayed as crosses.
    }
\end{figure}

Consider Figure~\ref{figure:large-inertia-individuals} where the individuals
have been generated with the same parameters as previously except with adjusted
row limits: \(R = (50, 100)\). In this case, the worst individuals are
equivalent, and the best-performing clusters are still dense about a single
point despite the trivial case being removed from consideration. Perhaps then,
this compact clustering is ``optimal''. However, more extensive studying may be
done. That is, the defined fitness function may require further attention.

Indeed, the final inertia could be considered a flawed or fragile fitness
function if it is supposed to evaluate the appropriateness or efficacy of the
\(k\)-means algorithm. Incorporating the inter-cluster spread to the fitness of
an individual dataset can reduce this observed compaction. The silhouette
coefficient is a metric used to evaluate the appropriateness of a clustering to
a dataset, and is given by the mean of the silhouette value, \(S(x)\), of each
point \(x \in Z_j\) in each cluster:

\begin{equation}
    \begin{gathered}
        A(x) := \frac{1}{|Z_j| - 1} \sum_{y \in Z_j \setminus \{x\}} d(x, y),
        \qquad B(x) := \min_{k \neq j} \frac{1}{|Z_k|} \sum_{w \in Z_k} d(x, w)
        \\\\
        S(x) := 
            \begin{cases}
                \frac{B(x) - A(x)}{\max\left\{A(x), B(x)\right\}}
                &\quad \text{if } |Z_j| > 1\\
                0 &\quad \text{otherwise}
            \end{cases}\label{eq:silhouette}
    \end{gathered}
\end{equation}\\

The optimisation of the silhouette coefficient is analogous to finding a dataset
which increases both the intra-cluster cohesion (the inverse of \(A\)) and
inter-cluster separation (\(B\)). Hence, the inertia is addressed by maximising
cohesion. Meanwhile, the spread of the clusters themselves is considered by
maximising separation.

Repeating the trials with the same parameters as previously, the silhouette
fitness function yields the results summarised in
Figure~\ref{figure:silhouette}. In this case, the datasets produced have reduced
overlap with one another whilst maintaining low values in the final inertia of
the clustering as shown in Figure~\ref{figure:silhouette-individuals}. Again,
the form of the individual clusters is much the same. The low values of inertia
correspond to tight clusters, and the tightest clusters are those with a minimal
number of points, i.e.\ a single point. As with the previous example, albeit at
a much slower rate, the preferable individuals are those leading toward this
case. That this gradual reduction in the dimension of the individuals occurs
after the improvement of the fitness function bolsters the claim that the base
case is also optimal.

However, due to the nature of the implementation, any individual from any
generation may be retrieved and studied should the final results be too
concentrated on the base case. This transparency in the history and progression
of the proposed method is something that sets it apart from other methods of the
same ilk such as GANs which have a reputation of providing so-called ``black
box'' solutions.

\begin{figure}[htbp]
    \ContinuedFloat*
    \centering
    \begin{minipage}{\imgwidth}
        \centering
        \includegraphics[width=\linewidth]{img/silhouette-fitness.pdf}
    \end{minipage}

    \begin{minipage}{\imgwidth}
        \centering
        \includegraphics[width=\linewidth]{img/silhouette-nrows.pdf}
    \end{minipage}
    \caption{Progression for silhouette and dimension across 1000 epochs at 100
             epoch intervals with \(R~=~(3,100)\).}\label{figure:silhouette}
\end{figure}

\begin{figure}[htbp]
    \ContinuedFloat%
    \centering
    \begin{minipage}{\imgwidth}
        \centering
        \includegraphics[width=\linewidth]{img/large-silhouette-fitness.pdf}
    \end{minipage}

    \begin{minipage}{\imgwidth}
        \centering
        \includegraphics[width=\linewidth]{img/large-silhouette-nrows.pdf}
    \end{minipage}
    \caption{Progression for silhouette and dimension across 1000 epochs at 100
             epoch intervals with
             \(R~=~(50,100)\).}\label{figure:large-silhouette}
\end{figure}

\begin{figure}[htbp]
    \ContinuedFloat*
    \centering
    \includegraphics[width=\imgwidth]{img/silhouette-individuals-0.pdf}
    \caption{Representative individuals based on silhouette with
             \(R~=~(3,100)\). Centroids displayed as
             crosses.}\label{figure:silhouette-individuals}
\end{figure}

\begin{figure}[htbp]
    \ContinuedFloat%
    \centering
    \includegraphics[width=\imgwidth]{img/large-silhouette-individuals-0.pdf}
    \caption{Representative individuals based on silhouette with
             \(R~=~(50,100)\). Centroids displayed as
             crosses.}\label{figure:large-silhouette-individuals}
\end{figure}


\subsection{Comparison with DBSCAN}

The capabilities of EDO as a tool for understanding an algorithm are highlighted
particularly when comparing an algorithm against another (or set of others)
simultaneously. This is done by utilising the freedom of choice in a fitness
function for EDO.\ Consider two algorithms, \(A\) and \(B\), and some common
metric between them, \(g\). Then understanding their similarities and contrasts
can be done by considering the differences in this metric on the two algorithms.
In terms of EDO, this means using \(f = g_A - g_B\), \(f = g_B - g_A\) or \(f
= \left| g_B - g_A \right|\) as the fitness function. By doing so, pitfalls,
edge cases or fundamental conditions for the method can be highlighted.
Overall, this process allows the researcher to more deeply learn about the
method of interest.

As an example of this process, consider the another clustering algorithm of a
different form such as Density Based Spatial Clustering of Applications with
Noise (DBSCAN) and suppose the objective is to find datasets for which
\(k\)-means outperforms this alternative. Here there is no concept of inertia as
DBSCAN is density-based and is able to identify outliers~\cite{Ester1996}. As
such, a valid (and arguably more appropriate) metric is the silhouette score as
defined in~(\ref{eq:silhouette}).

However, an adjustment to the fitness function must be made so as to accommodate
for the condition of the silhouette coefficient that there be more than one
cluster present. Let \(S_k (X)\) and \(S_D (X)\) denote the silhouette
coefficients of the clustering found by \(k\)-means and DBSCAN respectively.
Then the fitness function is defined to be:
\begin{equation}
    f(X) = 
        \begin{cases}
            S_D (X) - S_k (X), &\quad \text{if DBSCAN identifies two or more
            clusters (including noise)}\\
            \infty &\quad \text{otherwise.}
        \end{cases}\label{equation:dbscan-fitness}
\end{equation}

There are two remarks here. First, note the order of the subtraction here as EDO
minimises fitness functions by default. Also, \(f\) takes values in the range
\([-2, 2]\) where \(-2\) is the best, i.e.\ \(S_D(X) = -1\) and \(S_k(X) = 1\).
Likewise, 2 is the worst score.

It must also be acknowledged that \(k\)-means and DBSCAN share no common
parameters and so direct comparison is more difficult. For the purposes of this
example, only one set of parameters is used but a thorough investigation should
include a parameter sweep in cases such as these. The parameters being used are
\(k~=~3\) for \(k\)-means, and \(\epsilon~=~0.1,\ MinPoints~=~5\) for DBSCAN.\
This set was chosen following informal experimentation using the Python library
Scikit-learn~\cite{scikit} to find comparable parameters in the given search
space defined by the EDO parameters used previously with \(R~=~(50,100)\).

\begin{figure}[htbp]
    \centering
    \begin{minipage}{\imgwidth}
        \centering
        \includegraphics[width=\linewidth]{img/dbscan-fitness.pdf}
    \end{minipage}

    \begin{minipage}{\imgwidth}
        \centering
        \includegraphics[width=\linewidth]{img/dbscan-nrows.pdf}
    \end{minipage}
    \caption{Progressions for difference in silhouette (\(k\)-means-preferable)
             and dimension across 1000 epochs at 100 epoch
             intervals.}\label{figure:dbscan-silhouette}
\end{figure}

\begin{figure}[htbp]
    \centering
    \includegraphics[width=\imgwidth]{img/kmeans-individuals-0.pdf}

    \includegraphics[width=\imgwidth]{img/dbscan-individuals-0.pdf}
    \caption{Representative individuals with clusters shown. Convex hull shown
             as outline, and approximate concave hull shaded. Above by
             \(k\)-means, below by DBSCAN (noise points shown with small
             markers).}\label{figure:dbscan-individuals}
\end{figure}

Figure~\ref{figure:dbscan-silhouette} shows a summary of the progression of EDO
for this use case. As with the previous examples where \(R~=~(50, 100)\), the
variation in the population fitness is unstable but there is a clear trend of
improvement in the best individual over the course of the run. There is also a
convergence seen in the number of rows a dataset has. The resting dimension
varied across the trials conducted in this work but none exhibited a shift
toward the lower limit of 50 rows as with previous examples. This is suggestive
of a more competitive environment for individuals where slight changes to an
individual can drastically alter their fitness.

The effect of such changes can be seen in Figure~\ref{figure:dbscan-individuals}
where representative individuals are shown for this example. Here, the best
performing individual, when clustered by \(k\)-means, shows three clear and
nicely separated clusters. Note that they are not so tightly packed; again, this
suggests that the route to an optimal individual is less clearly defined. In
contrast, when the same dataset is clustered by DBSCAN a single cluster is found
with a single noise point held within the convex hull of the cluster, i.e.\
there are overlapping clusters (since noise points form a single cluster).
Hence, along with the fact that the larger cluster is widely spread, it follows
that the clustering have a relatively small, negative silhouette coefficient.

Another point of interest here is the convexity of the clusters. One of the
conditions for the success of \(k\)-means is that it requires convex clusters as
the objective is to approximate the centroidal Voronoi
tessellation~\cite{Du2006}. Without this condition, up to the correct choice of
\(k\), the algorithm will fail to produce adequate results for either inertia or
silhouette. DBSCAN, however, does not have this condition and is able to detect
non-convex clusters so long as they are dense enough.
Figure~\ref{figure:dbscan-individuals} shows the convex and concave hulls of the
clusters found by each method. The ``concave hull'' of a cluster is taken to be
the \(\alpha\)-shape of the cluster's data points~\cite{Edelsbrunner1983} where
\(\alpha\) is determined to be the smallest value such that all the points in
the cluster are contained in a single polygon. The convexity of cluster \(Z_j\),
denoted \(\mathcal{C}_j\), is then determined to be the ratio of the area of its
concave hull, \(H_c\), to the area of its convex hull, \(H_v\)~\cite{Sonka1993}:

\begin{equation}
    \mathcal{C}_j := \frac{area(H_c)}{area(H_v)}
\end{equation}

With this definition, it should be clear that a perfectly convex cluster would
have \(C_j = 1\).

It can be seen that the convexity of the clustering found by \(k\)-means appears
to be higher than that by DBSCAN.\ This was apparent across all trials conducted
in this work and indicates that the condition for convex clusters is being
sought out by during the optimisation process. Meanwhile, however, it is not
clear whether the performance of DBSCAN falls owing to its parameters or the
method itself. This is a point where parameter sweeping would prove most useful
so as to determine a crossing point for these two driving forces.

% TODO Perhaps some statistical test could be used here, or with the second part
% of this example.

Now, to bolster the discussion above, it is required that the inverse
optimisation be considered. That is, using the same parameters, investigate the
datasets for which DBSCAN outperforms \(k\)-means with respect to the silhouette
coefficient. This is equivalent to using \(-f\) as the fitness function except
with the same penalty of \(\infty\) for the case set out
in~(\ref{equation:dbscan-fitness}).
Figures~\ref{figure:negative-dbscan}~\&~\ref{figure:negative-dbscan-individuals}
show the equivalent summary as above with the revised fitness function.

\begin{figure}[htbp]
    \centering
    \begin{minipage}{\imgwidth}
        \centering
        \includegraphics[width=\linewidth]{img/negative-dbscan-fitness.pdf}
    \end{minipage}

    \begin{minipage}{\imgwidth}
        \centering
        \includegraphics[width=\linewidth]{img/negative-dbscan-nrows.pdf}
    \end{minipage}
    \caption{Progressions for difference in silhouette (DBSCAN-preferable) and
             dimension across 1000 epochs at 100 epoch
             intervals.}\label{figure:negative-dbscan}
\end{figure}

\begin{figure}[htbp]
    \centering
    \includegraphics[width=\imgwidth]{img/negative-kmeans-individuals-0.pdf}

    \includegraphics[width=\imgwidth]{img/negative-dbscan-individuals-0.pdf}
    \caption{Representative individuals with clusters shown. Convex hull shown
             as outline, and approximate concave hull shaded. Above by
             \(k\)-means, below by DBSCAN (noise points shown with small
             markers).}\label{figure:negative-dbscan-individuals}
\end{figure}

\section{Conclusion}

In this paper we have introduced a novel approach to understanding the quality
of an algorithm by exploring the space in which their well-performing datasets
exist. Following a detailed explanation of its internal mechanisms, a case study
in \(k\)-means clustering was offered as validation for the method. The method
utilises biological operators to traverse the space of all possible datasets in
an organic, unrestricted way. The generative nature of the proposed method also
provides transparency and richness to the solution when compared to other
contemporary techniques for artificial data generation as entire history of
individuals is preserved.

The evolutionary dataset optimisation method is dependent on a number of
parameters set out in this paper and perhaps the most important of which is the
choice of distribution families, \(\mathcal{P}\); these families set out the
general statistical shape of the columns of the datasets that are produced and
also control the present data types. The relationship between columns and their
associated distribution is not causal and appropriate methods should be
employed to understand the structure and characteristics of the data produced
before formal conclusions are made as set out in the example provided.


\pagebreak\appendix
\section{Appendix}

\subsection{Lloyd's algorithm}\label{appendix:kmeans}

\balg%
\KwIn{a dataset \(X\), a number of centroids \(k\), a distance metric \(d\)}
\KwOut{a partition of \(X\) into \(k\) parts, \(Z\)}

\Begin{%
    select \(k\) initial centroids, \(z_1, \ldots, z_k \in X\)\;
    \While{any point changes cluster or some stopping criterion is not met}{%
        assign each point, \(x \in X\), to cluster \(Z_{j^*}\) where:
        \[
            j^* = \argmin_{j = 1, \ldots, k} \left\{%
                {d\left(x, z_j\right)}^2
            \right\}
        \]\;
        recalculate all centroids by taking the intra-cluster mean:
        \[
            z_j = \frac{1}{|Z_j|} \sum_{x \in Z_j} x
        \]
    }
}
\caption{\(k\)-means (Lloyd's)}\label{algorithm:kmeans}
\ealg%


\subsection{Implementation example}\label{appendix:code}

Below is an example of how the Python implementation was used to complete the
first example, including the definition of the fitness function.

\singlespacing\begin{minted}{python}
import edo
from edo.pdfs import Uniform
from sklearn.cluster import KMeans


def fitness(dataframe, seed):
    """ Return the final inertia of 2-means on dataframe. """

    km = KMeans(n_clusters=2, random_state=seed).fit(dataframe)
    return km.inertia_


Uniform.param_limits["bounds"] = [0, 1]
row_limits, col_limits = [3, 100], [2, 2]

edo.run_algorithm(
    fitness,
    size,
    row_limits,
    col_limits,
    families=[Uniform],
    max_iter=1000,
    best_prop=selection,
    mutation_prob=mutation,
    seed=seed,
    root="out",
    fitness_kwargs={"seed": seed},
)
\end{minted}

\printbibliography%
\end{document}
