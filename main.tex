\documentclass[10pt]{article}

\usepackage[margin=1in, includefoot, footskip=30pt]{geometry}


\usepackage[ruled,vlined]{algorithm2e}
\usepackage{amssymb}
\usepackage{amsmath}
\usepackage{amsthm}
\usepackage{booktabs}
\usepackage{caption}
\usepackage{graphicx}
\usepackage{hyperref}
\usepackage{setspace}
\usepackage{standalone}
\usepackage{tikz}
    \usetikzlibrary{%
        arrows.meta,
        decorations.pathreplacing,
        decorations.text,
        patterns,
        shapes.arrows,
        shapes.geometric
    }


% Page setup and lengths
\captionsetup{font=doublespacing}
\doublespacing%

\newlength{\imgwidth}
\setlength{\imgwidth}{0.9\textwidth}

\definecolor{magenta}{RGB}{226, 62, 138}


% Algorithms
\newcommand{\balg}{\begin{algorithm}[htbp]\singlespacing\DontPrintSemicolon}
\newcommand{\ealg}{\end{algorithm}\doublespacing}


% TikZ styles, commands and settings
\tikzstyle{every picture} += [remember picture]
\tikzstyle{na} = [baseline=-.5ex]

\tikzset{%
    column/.pic={%
        code{%
            \draw[line width=1pt] (0, 0) rectangle (-2cm, 4cm);
            \foreach \val in {0, ..., #1}{%
                \draw[rotate=90] ([xshift=-\val*10pt] 4cm, 2cm) -- ++(0, -2cm);
            };
            \node at (-1cm, 1.25) {$\vdots$};
            \foreach \val in {1, 2}{%
                \draw (0, \val * 10pt) -- ++(-2cm, 0);
            };
        }
    }
}

\tikzset{%
    fullcolumn/.pic={%
        code{%
            \draw[line width=1pt] (0, 0) rectangle (-2cm, #1*10pt);
            \foreach \val in {0, ..., #1}{%
                \draw[rotate=90] ([xshift=-\val*10pt] #1*10pt, 2cm) -- ++(0, -2cm);
            };
        }
    }
}

\newcommand{\inputtikz}[3][\imgwidth]{%
    \begin{figure}[htbp]
        \centering
        \resizebox{#1}{!}{%
            \input{tex/#2}
        }
        \caption{#3}
    \end{figure}
}


% Document
\title{%
    Evolutionary Dataset Optimisation:
    learning algorithm quality through genetic evolution
}
\author{Henry Wilde, Vincent Knight, Jonathan Gillard}
% TO DO: Finalise order and preferred name
\date{}

\begin{document}

\maketitle%
\bigskip%
\begin{abstract}
    When faced with a problem involving data, it is almost certainly the case
    that the data is fixed and in order to do something with that data, a
    researcher must select an algorithm that is appropriate for the problem
    domain whilst performing well on their data. The value prescribed to an
    algorithm is often found through a process of surveying the current
    literature to create a shortlist, then running various trials with the
    shortlisted algorithms. The winning algorithm is then chosen based on some
    common objective value. The issue with this process is that it does not
    necessarily allow (or require) the researcher to consider why certain
    algorithms perform better on particular datasets and not others, and which
    characteristics make data ``good'' for their chosen algorithm.

    This paper introduces a novel method for generating artificial data through
    genetic evolution, the purpose of which is to create populations of datasets
    for which a particular algorithm performs well. This is done by passing an
    algorithm's objective function to an evolutionary algorithm. Therein, each
    individual is a particular dataset defined by its dimensions, entries, and
    the approximate statistical shape of each of its attributes. In this way,
    detailed information about each individual is retained throughout the
    algorithm. Hence, they may be manipulated in a meaningful way during the
    run, and studied once the algorithm has terminated.

    Following this, a number of examples are given to show the performance of
    the method. These examples are created using a Python implementation of the
    process which is built to be highly customisable, interpretable and
    reproducible.
\end{abstract}

\newpage%
%=================================================
\section{Introduction}\label{section:introduction}

\inputtikz{flowchart.tex}{%
    A general schematic for an evolutionary algorithm.
}

\begin{itemize}
    \item What is the motivation?
    \item What is the problem?
    \item What is the solution?
\end{itemize}

%-----------------------------
\subsection{Literature review}

\begin{itemize}
    \item How is artificial data made?
    \item Why hasn't this been done before?
    \item Genetic algorithms used to train algorithms for data
    \item Diagram showing that this is the ``reverse'' problem
\end{itemize}


%============================================================
\section{The evolutionary algorithm}\label{section:algorithm}

%---------------------
\subsection{Structure}

\begin{itemize}
    \item Algorithm statement with components
    \item Discussion on the choice to include custom stopping/dwindling
        conditions, compacting the search space, etc.
    \item Operators and detailed mechanisms to come later.
\end{itemize}

\balg%
\KwData{%
    \begin{tabular}{l}
    Fitness function, \(f: X \to \mathbb{R}\);
    population size, \(N\);
    row limits, \(R = [r_l, r_u]\);
    column limits, \(C = [c_l, c_u]\);\\

    Column distributions, \(P\);
    relative weights for \(P\), \(w \in {[0, 1]}^{|P|}\) s.t.\ \(\sum w = 1\);
    maximum iterations, \(M\);\\

    Best proportion, \(\delta\);
    lucky proportion, \(\epsilon\);
    mutation probability, \(\mu\);
    compaction ratio, \(s\).
    \end{tabular}
}
\KwResult{A full history of the populations and their fitnesses.}\;

\textcolor{blue}{initialisation}\;
\(population \longleftarrow CreateInitialPopulation(N,R,C,P,w)\)\;
\(populationFitness \longleftarrow GetPopulationFitness(population,f)\)\;
\(%
    populationHistory \longleftarrow%
    \left\{(population, populationFitness)\right\}
\)\;
\(i \longleftarrow 0\)\;\;

\textcolor{blue}{begin iterative step}\;
\While{\(iteration < M\) and not \textbf{STOP}}{%
    \textcolor{blue}{select parent individuals}\;
    \(%
        parents \longleftarrow%
        Selection(population,populationFitness,\delta,\epsilon)
    \)\;

    \textcolor{blue}{create new population}\;
    \(population \longleftarrow CreateNewPopulation(parents,N,R,C,P,w,\mu)\)\;

    \textcolor{blue}{update fitness and history}\;
    \(populationFitness \longleftarrow GetPopulationFitness(population,f)\)\;
    \(%
        populationHistory \longleftarrow%
        populationHistory \cup \left\{(population,populationFitness)\right\}
    \)\;

    \textcolor{blue}{adjust parameters}\;
    \(i \longleftarrow i + 1\)\;
    \(P \longleftarrow ReduceMutationSpace(parents,P,i,M,s)\)\;
    \(\mu, \textbf{STOP} \longleftarrow UpdateParameters\)\;
}
\caption{The evolutionary dataset optimisation algorithm}
\ealg%


\balg%
\KwData{\(N,R,C,P,w\)}
\KwResult{A random population of datasets of size \(N\), \(population\)}\;

\(population \longleftarrow \emptyset\)\;
\For{\(i\leftarrow 1 \ \KwTo \ N\)}{%
    \(%
        population \longleftarrow%
        population \cup \left\{CreateIndividual(R,C,P,w)\right\}
    \)\;
}
\caption{\(CreateInitialPopulation\)}
\ealg%

\balg%
\KwData{\(population,f\)}
\KwResult{Fitness of each individual in the population, \(populationFitness\)}\;

\(populationFitness \longleftarrow \emptyset\)\;
\For{\(individual \in population\)}{%
    \(%
        populationFitness \longleftarrow%
        populationFitness \cup \left\{f(individual)\right\}
    \)\;
}
\caption{\(GetPopulationFitness\)}
\ealg%

\balg%
\KwData{\(parents,N,R,C,P,w,\mu\)}
\KwResult{A new population of size \(N\), \(newPopulation\)}\;

\(newPopulation \longleftarrow parents\)\;
\While{\(|newPopulation| < N\)}{%
    Sample two parents, \(parentOne, parentTwo\), from \(parents\)\;
    \(offspring \longleftarrow Crossover(parentOne,parentTwo,C,P,)\)\;
    \(mutant \longleftarrow Mutation(offspring,\mu,R,C,P,w)\)\;
    \(newPopulation \longleftarrow newPopulation \cup \{mutant\}\)\;
}
\caption{\(CreateNewPopulation\)}
\ealg%

\balg%
\KwData{}
\KwResult{}\;

\textcolor{blue}{initialisation}\;
\(S \longleftarrow 1 - \frac{i}{s M}\)\;
\(currentParameters \longleftarrow GetCurrentParameters(parents, P)\)\;\;

\textcolor{blue}{compute new bounds on distribution parameters}\;
\For{\(parameters \in currentParameters\)}{%
    \For{\(parameter, parameterValues \in parameters\)}{%
        \(m \longleftarrow Mean(parameterValues)\)\;
        \(%
            \Delta \longleftarrow%
            \frac{S}{2} \times (\max parameterValues - \min parameterValues)
        \)
        \(%
            L \longleftarrow%
            \max\{parameterLimits[0], \min\{m - \Delta, m + \Delta\}\}
        \)\;
        \(%
            U \longleftarrow%
            \min\{parameterLimits[1], \max\{m - \Delta, m + \Delta\}\}
        \)\;
        \(parameterLimits \longleftarrow [L, U]\)\;
    }
}
\caption{\(ReduceMutationSpace\)}
\ealg%

\balg%
\(%
    currentParameters \longleftarrow%
    \left(p: \left(\emptyset, \ldots, \emptyset\right) | \ p \in P\right)
\)\;
\For{\(parent \in parents\)}{%
    \For{each column in \(parent\)}{%
        \(p \longleftarrow\) distribution class of column\;
        \If{no entry for \(p\) in \(currentParameters\)}{%
            \(currentParameters[p] \longleftarrow \emptyset\)
        }
        \For{\((parameter, parameterValue) \in p\)}{%
            \(%
                currentParameters[p] \longleftarrow%
                currentParameters[p] \cup%
                \left\{(parameter, parameterValue)\right\}
            \)\;
        }
    }
}
\caption{\(GetCurrentParameters\)}
\ealg%

Additional algorithms for the individual creation and operator processes are
given in Section~\ref{subsection:mechanisms}. However, there is no defined
process for how to stop the algorithm or adjust the mutation probability,
\(\mu\). This is deliberate as such conditions are specific to the problem
domain. As such, in the Python implementation, a user may define their stopping
condition or \(\mu\)-adjustment to be any function. Some examples include:

\begin{itemize}
    \item These should be real-world examples used in other EA with references.
    \item Stopping when the difference in the variance of the current and last
        population fitnesses is below some tolerance.
    \item Halving the mutation probability every 100 iterations.
\end{itemize}
        

%------------------------------------------------------------
\subsection{Internal mechanisms}\label{subsection:mechanisms}

\subsubsection{Individuals}

\inputtikz{individual.tex}{%
    An example of how an individual is first created.
}

\subsubsection{Selection}

\inputtikz{selection.tex}{%
    An example of the selection process with the inclusion of some lucky
    individuals.
}

\subsubsection{Crossover}

\inputtikz{crossover.tex}{An example of the crossover process.}

\subsubsection{Mutation}

\inputtikz{mutation.tex}{An example of the mutation process.}

\subsubsection{Shrinking}

\inputtikz{shrinking.tex}{An diagram of the shrinking process.}

%-----------------------------------------------------------
\subsection{Implementation}\label{subsection:implementation}

\begin{itemize}
    \item Documentation: \url{https://edo.readthedocs.io}
    \item Repo: \url{https://github.com/daffidwilde/edo}
\end{itemize}


%===================================================
\section{Numerical examples}\label{section:examples}

\begin{itemize}
    \item The \(x^2\) example from the docs is a nice easy one to illustrate
        things
    \item Something stochastic
    \item Make use of the moving parts (linear focus of the mutation space,
        dwindling mutation probability, stopping conditions)
    \item If not the \(k\)-modes initialisation example, then another clustering
        one. Maybe \(k\)-means versus DBSCAN to show DBSCAN works better on
        non-convex clusters.
\end{itemize}

\end{document}
