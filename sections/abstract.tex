\begin{abstract}
    In this paper we propose a new method for learning how algorithms perform.
    Classically, algorithms are compared on a finite number of existing (or
    newly simulated) benchmark data sets based on some fixed metric. The
    algorithm(s) with the smallest value of this metric are chosen to be the
    `best performing'. We offer a new approach to flip this paradigm. We instead
    aim to gain a richer picture of the performance of an algorithm by
    generating artificial data through genetic evolution, the purpose of which
    is to create populations of datasets for which a particular algorithm
    performs well. These data sets can be studied to learn as to what attributes
    lead to a particular progress of a given algorithm.

    Following a detailed description of the algorithm as well as a brief
    description of an open source implementation, a number of numeric
    experiments are presented to show the performance of the method.
\end{abstract}
