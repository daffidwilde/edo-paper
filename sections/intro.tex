\section{Introduction}\label{section:introduction}

This work presents a novel approach to learning the quality and performance of
an algorithm through the use of evolution. When an algorithm is developed to
solve a given problem, the designer is presented with questions about the
performance of their proposed method, and its relative performance against
existing methods. This is an inherently difficult task. However, under the
current paradigm, the standard response to this situation is to use a known
fixed set of datasets and a common metric amongst the proposed method and its
competitors. The algorithm is then assessed based on this metric with minimal
consideration for both the appropriateness or reliability of the datasets being
used, and the robustness of the method in question.

This notion is not so easily observed when travelling in the opposite direction.
Suppose that, instead, the benchmark was a dataset of particular interest and a
preferable algorithm was to be determined for some task. There exist a number of
methods employed across disciplines to complete this task that take into account
the characteristics of the data and the context of the research problem. These
methods include the use of diagnostic tests. For instance, in the case of
clustering, if the data displayed an indeterminate number of non-convex blobs,
then one could recommend that an appropriate clustering algorithm would be
DBSCAN~\cite{Ester1996}. Otherwise, for scalability, \(k\)-means may be
chosen.

The approach presented in this work aims to flip the paradigm described here by
allowing the data itself to be unfixed. This fluidity in the data is achieved by
generating data for which the algorithm performs well (or better than some
other) through the use of an evolutionary algorithm. The purpose of doing so is
not to simply create a bank of useful datasets but rather to allow for the
subsequent studying of these datasets. In doing so, the attributes and
characteristics which lead to the success (or failure) of the algorithm may be
described, giving a broader understanding of the algorithm on the whole.

\inputtikz{paradigm}{%
    On the right: the current path for selecting some algorithm(s) based on
    their validity and performance for a given dataset. On the left: the
    proposed flip to better understand the space in which `good' datasets exist
    for an algorithm.
}

This proposed flip has a number of motivations, and below is a non-exhaustive
list of some of the problems that are presented by the established evaluation
paradigm:

% TODO More citations required here. Places to start looking: Hyndman blogpost,
% open science papers, p-hacking critique, data carpentry slides.

\begin{enumerate}
    \item How are these benchmark examples selected? There is no true measure of
        their reliability other than their frequent use. In some domains and
        disciplines there are well-established benchmarks so those found through
        literature may well be reliable, but in others less so.
    \item Sometimes, when there is a lack of benchmark examples, a `new' dataset
        is simulated to assess the algorithm. This begs the question as to how
        and why that simulation is created. Not only this, but the origins of
        existing benchmarks is often a matter of convenience rather than their
        merit.
    \item In disciplines where there are established benchmarks, there may still
        be underlying problems around the true performance of an algorithm:
        \begin{enumerate}[(i)]
            \item As an example, work by Torralba and Efros~\cite{Torralba2011}
                showed that image classifiers trained and evaluated on a
                particular dataset, or datasets, did not perform reliably when
                evaluated using other benchmark datasets that were determined
                to be similar. Thus leading to a model which lacks robustness.
            \item The amount of learning one can gain as to the characteristics
                of data which lead to good (or bad) performance of an algorithm
                is constrained to the finite set of attributes present in the
                benchmark data chosen in the first place.
        \end{enumerate}
\end{enumerate}

Evolutionary algorithms (EAs) have been applied successfully to solve a wide
array of problems \-- particularly where the complexity of the problem or its
domain are significant. These methods are highly adaptive and
their population-based construction (displayed in Figure~\ref{figure:flowchart})
allows for the efficient solving of problems that are otherwise beyond the scope
of traditional search and optimisation methods.

\inputtikz{flowchart}{%
    A general schematic for an evolutionary algorithm.
}

The use of EAs to generate artificial data is not a new concept. Its
applications in data generation have included developing methods for the
automated testing of software~\cite{Koleejan2015,Michael2001,Sharifipour2018}
and the synthesis of existing or confidential data~\cite{Chen2016}. Such methods
also have a long history in the parameter optimisation of algorithms, and
recently in the automated design of convolutional neural network (CNN)
architecture~\cite{Suganuma2017,Sun2018}.

Other methods for the generation or synthesis of artificial data include
simulated annealing~\cite{Matejka2017} and generative adversarial networks
(GANs)~\cite{Goodfellow2014}. The unconstrained learning style of methods such
as CNNs and GANs aligns with that proposed in this work. By allowing the EA to
explore and learn about the search space in an organic way, less-prejudiced
insight can be established that is not necessarily reliant on any particular
framework or agenda.

Note that the proposed methodology is not simply to use an EA to optimise an
algorithm over a search space with fixed dimension or datatype such as those set
out in~\cite{Chen2016}. The size and sample space itself is considered as a
property that can be traversed through the algorithm.
