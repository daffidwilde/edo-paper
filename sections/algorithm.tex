\section{The evolutionary algorithm}\label{section:algorithm}

\subsection{Structure}

On the surface, the evolutionary algorithm in EDO follows a fairly generic
schema; there is a random initial population, individuals are selected through a
modified truncation process, and a uniform crossover operator is used. However,
there are some additional features and the processes within are designed with
the objective of artificial data generation in mind. With that, there are a
number of parameters passed to the algorithm, including several that are not
typical:

\begin{itemize}
    \item A real-valued fitness function, \(f: X \to \mathbb{R}\), which acts on
        a dataset to return a single fitness score.
    \item A population size, \(N \in \mathbb{N}\).
    \item Limits on the number of rows a dataset can have,
        \(R \in \left\{(l, u) \in \mathbb{N}^2~|~l \leq u\right\}\).
    \item Limits on the number of columns a dataset can have,
        \(C \in \left\{(l, u) \in \mathbb{N}^2~|~l \leq u\right\}\). Note that
        it is possible to use a different form for \(C\) so as to specify a
        minimum or maximum number of columns a dataset should have for each
        distribution passed in \(\mathcal{P}\).
    \item A pool of probability distribution families, \(\mathcal{P}\). Each
        family in this pool has a set of parameter limits which form a part of
        the overall search space. For instance, the normal distribution family,
        denoted by \(N(\mu, \sigma^2)\), would have limits on reasonable values
        for the mean, \(\mu\), and the standard deviation, \(\sigma\).
    \item A probability vector to sample distributions from \(\mathcal{P}\),
        \(w = \left(w_1, \ldots, w_{|\mathcal{P}|}\right)\) such that
        \(\sum_{i=1}^{|\mathcal{P}|} w_i = 1\) and
        \(w_i \in [0, 1] \ \forall~i = 1, \ldots, |\mathcal{P}|\).
    \item A maximum number of iterations, \(M \in \mathbb{N}\).
    \item Two selection parameters: one to indicate the proportion of the 
        fittest individuals to carry forward, \(b \in [0, 1]\), and to allow for
        a small proportion of ``lucky'' individuals in the next generation,
        \(l \in [0, 1]\).
    \item A mutation probability, \(p_m \in [0, 1]\), used to indicate the
        intensity and likelihood of the mutation a new individual might undergo.
    \item A shrink factor, \(s \in [0, 1]\). The relative size of a component of
        the search space to be retained after adjustment.
\end{itemize}

The concepts discussed in this section form the mechanisms of the evolutionary
dataset optimisation algorithm. To use the algorithm practically, these
components have been implemented in Python as a library built on the scientific
Python stack (NumPy and Pandas). The library is fully documented (documentation
at \url{https://edo.readthedocs.io}) and is available online~\cite{edo-project}.

\balg%
\KwData{\(f, N, R, C, \mathcal{P}, w, M, b, l, \mu, s\)}
\KwResult{A full history of the populations and their fitnesses.}

\Begin{%
    create initial population of individuals\;
    find fitness of each individual\;
    record population and its fitness\;

    \While{%
        current iteration less than the maximum
        \textbf{and} stopping condition not met
    }{%
        select parents based on fitness and selection proportions\;
        use parents to create new population through crossover and mutation\;
        find fitness of each individual\;
        update population and fitness histories\;
        \If{adjusting the mutation probability}{%
            update mutation probability
        }
        \If{using a shrink factor}{%
            shrink the mutation space based on parents
        }
    }
}
\caption{The evolutionary dataset optimisation algorithm}
\ealg%
\label{alg:edo}
\balg%
\KwData{parents, \(N, R, C, \mathcal{P}, w, p_m\)}
\KwResult{A new population of size \(N\)}

\Begin{%
    add parents to the new population\;
    \While{the size of the new population is less than \(N\)}{%
        sample two parents at random\;
        create an offspring by crossing over the two parents\;
        mutate the offspring according to the mutation probability\;
        add the mutated offspring to the population\;
    }
}
\caption{Creating a new population}
\ealg%


The statement of the EDO algorithm is deliberately vague here. This is, in part,
because of the customisable nature of its build but also to lay out its general
structure from a high level perspective. Lower level discussion is provided
below where additional algorithms for the
individual creation, evolutionary operator and shrinkage processes are given
along with diagrams (where appropriate).

Note that there are no defined processes for how to stop the algorithm or adjust
the mutation probability, \(p_m\). These form two key areas of potential
customisation since such conditions can be specific to the problem domain. With
that, as is detailed in the Python implementation, a user may define their
stopping condition or \(p_m\)-adjustment to be any reasonable function. Some
examples include:

\begin{itemize}
    \item These should be real-world examples used in other EA with references.
    \item Stopping when the difference in the variance of the current and last
        population fitnesses is below some tolerance.
    \item Halving the mutation probability every 100 iterations.
\end{itemize}

\subsection{Individuals}

Evolutionary algorithms operate on succeeding populations of individuals often
coined as ``generations''. Typically, an individual would be encoded as a bit
string of a fixed length and treated as a chromosome-like object to be
manipulated. In EDO, as the objective is to generate datasets, there is no
encoding process. Instead, the datasets are manipulated directly so that the
biological operators can behave and be interpreted in a more meaningful way.

\inputtikz{individual}{%
    An example of how an individual is first created.
}

In a sense, a dataset is treated similarly to a classical bit string as the
primary components (loci) of the dataset are considered to be its columns. As is
seen in Figure~\ref{figure:individual}, an individual's creation is defined by
the random generation of its columns. A set of instructions on how to sample new
values for that column are recorded in the form of a probability distribution.
These distributions are sampled from a pool of distribution families which is
passed to the evolutionary algorithm along with the other parameters.

Obviously, users and interpreters of EDO should not be so quick to assume that
these pairs of distributions and columns are typical of their partner. That is,
that the columns are a reliable representative of the distribution associated
with them, or vice versa. A caveat to this statement: this is particularly true
of ``shorter'' datasets with a small number of rows, whereas confidence in the
pair could be given more liberally for ``longer'' datasets. In the case of the
latter, the column metadata can become more useful for casually analysing the
data which is generated. In any case, however, more direct and sophisticated
methods should be employed to understand the structure and characteristics of
the data before formal conclusions are made.

\balg%
\KwData{\(R, C, \mathcal{P}, w\)}
\KwResult{An individual defined by a dataset and some metadata}

\Begin{%
    sample a number of rows and columns\;
    create an empty dataset\;
    \For{each column in the dataset}{%
        sample a distribution from \(\mathcal{P}\)\;
        create an instance of the distribution\;
        fill in the column by sampling from this instance\;
        record the instance in the metadata
    }
}
\caption{Creating an individual}
\ealg%


\subsection{Selection}

The selection operator describes the process by which individuals are chosen
from the current population to generate the next. Almost always, the likelihood
of an individual being selected is determined by their fitness. This is because
the purpose of selection is to preserve favourable qualities and encourage some
homogeneity within future generations.

\inputtikz{selection}{%
    The selection process with the inclusion of some lucky individuals.
}

In EDO, a modified truncation selection method is used as can be seen in
Figure~\ref{figure:selection}. Standard truncation selection takes a fixed
number, \(n_b = \lceil bN\rceil\), of the fittest individuals in a population
and makes them the ``parents'' of the next. The modification is an optional
stage after the best individuals have been chosen. By passing some small \(l\)
to the evolutionary algorithm, a number of random individuals can be selected to
be carried forward. This number is given by \(n_l = \lceil lN \rceil\). It
should be noted that even with this modification, no individual may be selected
more than once in a single iteration but could potentially be present throughout
the entire run of the algorithm.

\balg%
\KwIn{population, population fitness, \(b\), \(l\)}
\KwOut{A set of parent individuals}

\Begin{%
    calculate \(n_b\) and \(n_l\)\;
    sort the population by the fitness of its individuals\;
    take the first \(n_b\) individuals and make them parents\;
    \If{there are any individuals left}{%
        take the next \(n_l\) individuals and make them parents\;
    }
}
\caption{The selection process}\label{alg:selection}
\ealg%


It has been found that, despite its efficiency as a selection
operator, truncation selection can lead an evolutionary algorithm to
converge prematurely to local optima~\cite{Jebari2013}. Hence, the ability to
include some randomly selected individuals is included to encourage diversity
throughout the run of the algorithm. This feature can be particularly useful in
more complex optimisation scenarios \-- or for larger populations where a
greater loss of diversity is seen in the selection process simply due to the
nature of truncation selection~\cite{Tatsuya2002}. As such, it should be used
sparingly so as not to dominate the selection process with unwanted randomness.

\subsection{Crossover}

Crossover is the operation of combining two individuals in order to create at
least one offspring. Often in classical evolutionary algorithms, the term
``crossover'' is taken literally: two bit strings are crossed at a point
to produce two new bit strings. Another popular method is uniform crossover
where the loci of a new individual are sampled uniformly from either parent
individual. This method is adapted to support dataset manipulation here by
sampling first a set of dimensions from the parents, and then inheriting entire
columns.

\inputtikz{crossover}{The crossover process.}

\balg%
\KwIn{Two parents}
\KwOut{An offspring made from the parents ready for mutation}

\Begin{%
    collate the columns and metadata from each parent in a pool\;
    sample each dimension from between the parents uniformly\;
    form an empty dataset with these dimensions\;
    \For{each column in the dataset}{%
        sample a column (and its corresponding metadata) from the pool\;
        \If{this column is longer than required}{%
            randomly select entries and delete them as needed 
        }
        \If{this column is shorter than required}{%
            sample new values from the metadata and append them to the column as
            needed
        }
        add this column to the dataset and record its metadata\;
    }
}
\caption{The crossover process}
\ealg%


\subsection{Mutation}

Mutation is used in evolutionary algorithms to encourage a broader exploration
of the search space. Traditional applications with chromosome representations
would mutate by run along the loci of an individual and ``switching'' the binary
value with some constant probability. Under this framework, as has been
discussed, an individual's columns are similar to traditional loci. However, in
the mutation phase multiple characteristics of an individual are susceptible to
being mutated beyond the columns themselves such as the dimensions of the
individual and their column metadata.

\inputtikz{mutation}{The mutation process.}

The mutation process, seen in Figure~\ref{figure:mutation} and defined in
Algorithm~\ref{algorithm:mutation},
is deliberately fine so that all aspects of an individual can be changed in a
meaningful yet fluid way. Each of the potential mutations occur with the same
probability, \(p_m\), but the way in which columns are maintained assure that
(assuming appropriate choices for \(f\) and \(\mathcal{P}\)) many mutations in
the metadata and the dataset itself will only result in some incremental change
in the individual's genetics and fitness overall \-- at least relatively so to,
say, a completely new individual.

\balg%
\KwData{An individual, \(p_m\), \(R\), \(C\), \(\mathcal{P}\), \(w\)}
\KwResult{A mutated individual}

\Begin{%
    sample a random number \(r \in [0, 1]\)\;
    \If{\(r < p_m\) and adding a row would not violate \(R\)}{%
        sample a value from each distribution in the metadata\;
        append these values as a row to the end of the dataset\;
    }
    sample a new \(r \in [0, 1]\)\;
    \If{\(r < p_m\) and removing a row would not violate \(R\)}{%
        remove a row at random from the dataset
    }
    sample a new \(r \in [0, 1]\)\;
    \If{\(r < p_m\) and adding a new column would not violate \(C\)}{%
        create a new column using \(\mathcal{P}\) and \(w\)\;
        append this column to the end of the dataset
    }
    sample a new \(r \in [0, 1]\)\;
    \If{\(r < p_m\) and removing a column would not violate \(C\)}{%
        remove a column (and its associated metadata) at random from the dataset
    }
    \For{each distribution in the metadata}{%
        \For{each parameter of the distribution}{%
            sample a random number \(r \in [0, 1]\)\;
            \If{\(r < p_m\)}{%
                sample a new value from within the distribution parameter
                limits\;
                update the parameter value with this new value
            }
        }
    }
    \For{each entry in the dataset}{%
        sample a random number \(r \in [0, 1]\)\;
        \If{\(r < p_m\)}{%
            sample a new value from the associated column distribution\;
            update the entry with this new value
        }
    }
}
\caption{The mutation process}\label{algorithm:mutation}
\ealg%


\subsection{Shrinking}

The potential benefits of adapting the search space of an EA has been
well-discussed in the domain of complex optimisation problems. During the
development of EDO, two methods were considered to do this. Each of these
methods relied on a law relating a successive generation's search space with its
predecessor. It should be noted that in both cases the methods were adapted to
conform with the choice to represent individuals as they are rather than as bit
strings.

The first was developed to be a two-part process based on a linear law with two
parameters: a shrink factor, \(s \in (0, 1)\), and the maximum number of
iterations \(M \in \mathbb{N}\)~\cite{Amirjanov2017}. The adapted method would
be as follows. For all iterations, \(t \geq sM\), no shrinking would take place.
However, for all previous iterations, the suggested method would take the
parents found during selection and act on each component of the search space.
That is, for each iteration, \(t < sM\), every component would have its lower
and upper limits, denoted by \(l_t\) and \(u_t\) respectively, adjusted so that
they are centred about the mean parent value, \(\mu\), and be such that:
\[
    u_{t+1} - l_{t+1} = (u_t - l_t) \left(1 - \frac{t}{sM}\right)
\]

More specifically, the adjusted limits would be calculated as follows:
\begin{align*}
    l_{t+1} = \max \left\{%
        l_t, \ \mu - \frac{1}{2} (u_t - l_t) \left(1 - \frac{t}{sM}\right)
    \right\}\\
    u_{t+1} = \min \left\{%
        u_t, \ \mu + \frac{1}{2} (u_t - l_t) \left(1 - \frac{t}{sM}\right)
    \right\}
\end{align*}

The second method was for use in a genetic algorithm which mapped weighted bit
strings to some pre-defined interval with lower and upper
bounds~\cite{Amirjanov2016}. The proposed method would rely on a power law with
a single parameter: some shrink factor, \(s \in [0, 1]\). Again, at each
iteration, the parents would be taken and every component's limits would be
adjusted so that they were centred about the mean parent value, \(\mu\), such
that:
\[
    u_{t+1} - l_{t+1} = (u_t - l_t) s^t
\]

The process by which the values of \(l_{t+1}\) and \(u_{t+1}\) would be found
are equivalent to the above but with a different shift term:
\begin{align}
    \label{eq:shrinking_lower}
    l_{t+1}&= \max \left\{l_t, \ \mu - \frac{1}{2} (u_t - l_t) s^t\right\}\\
    \label{eq:shrinking_upper}
    u_{t+1}&= \max \left\{u_t, \ \mu + \frac{1}{2} (u_t - l_t) s^t\right\}
\end{align}

From these brief definitions alone, these methods appear to be largely
indistinguishable. However, following a wider discussion around how EDO would
work for the general user, it was decided that the first method be rejected in
favour of the second. It was found that the second method having fewer
parameters was a particularly redeeming feature; this removed any hidden or
otherwise unwanted interactions between a parameter dedicated to the shrink
process, \(s\), and the maximum number of iterations, \(M\), which is often used
as a fallback stopping criterion in complex problem domains.

\balg%
\KwIn{parents, current iteration, \(\mathcal{P}, M, s\)}
\KwOut{A new mutation space focussed around the parents}

\Begin{%
    \For{each distribution subtype in \(\mathcal{P}\)}{%
        \For{each parameter of the distribution}{%
            get the current values for parameter over all parent columns\;
            find the mean of the current values\;
            find the new lower~(\ref{eq:shrinking_lower}) and
            upper~(\ref{eq:shrinking_upper}) bounds around the mean\;
            set the parameter limits\;
        }
    }
}
\caption{Shrinking the mutation space}\label{alg:shrinking}
\ealg%


