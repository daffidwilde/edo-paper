\section{The evolutionary algorithm}\label{section:algorithm}

\subsection{Structure}

\begin{itemize}
    \item Discussion on the choice to include custom stopping/dwindling
        conditions, compacting the search space, etc.
    \item Operators and detailed mechanisms to come later.
\end{itemize}

\balg%
\KwData{%
    \begin{tabular}{l}
    Fitness function, \(f: X \to \mathbb{R}\);
    population size, \(N\);
    row limits, \(R = [r_l, r_u]\);
    column limits, \(C = [c_l, c_u]\);\\

    Column distributions, \(P\);
    relative weights for \(P\), \(w \in {[0, 1]}^{|P|}\) s.t.\ \(\sum w = 1\);
    maximum iterations, \(M\);\\

    Best proportion, \(\delta\);
    lucky proportion, \(\epsilon\);
    mutation probability, \(\mu\);
    compaction ratio, \(s\).
    \end{tabular}
}
\KwResult{A full history of the populations and their fitnesses.}\;

\textcolor{blue}{initialisation}\;
\(population \longleftarrow CreateInitialPopulation(N,R,C,P,w)\)\;
\(populationFitness \longleftarrow GetPopulationFitness(population,f)\)\;
\(%
    populationHistory \longleftarrow%
    \left\{(population, populationFitness)\right\}
\)\;
\(i \longleftarrow 0\)\;\;

\textcolor{blue}{begin iterative step}\;
\While{\(iteration < M\) and not \textbf{STOP}}{%
    \textcolor{blue}{select parent individuals}\;
    \(%
        parents \longleftarrow%
        Selection(population,populationFitness,\delta,\epsilon)
    \)\;

    \textcolor{blue}{create new population}\;
    \(population \longleftarrow CreateNewPopulation(parents,N,R,C,P,w,\mu)\)\;

    \textcolor{blue}{update fitness and history}\;
    \(populationFitness \longleftarrow GetPopulationFitness(population,f)\)\;
    \(%
        populationHistory \longleftarrow%
        populationHistory \cup \left\{(population,populationFitness)\right\}
    \)\;

    \textcolor{blue}{adjust parameters}\;
    \(i \longleftarrow i + 1\)\;
    \(P \longleftarrow ReduceMutationSpace(parents,P,i,M,s)\)\;
    \(\mu, \textbf{STOP} \longleftarrow UpdateParameters\)\;
}
\caption{The evolutionary dataset optimisation algorithm}
\ealg%
\label{alg:edo}
\balg%
\KwData{\(N,R,C,P,w\)}
\KwResult{A random population of datasets of size \(N\), \(population\)}\;

\(population \longleftarrow \emptyset\)\;
\For{\(i\leftarrow 1 \ \KwTo \ N\)}{%
    \(%
        population \longleftarrow%
        population \cup \left\{CreateIndividual(R,C,P,w)\right\}
    \)\;
}
\caption{\(CreateInitialPopulation\)}
\ealg%


\balg%
\KwData{\(parents,N,R,C,P,w,\mu\)}
\KwResult{A new population of size \(N\), \(newPopulation\)}\;

\(newPopulation \longleftarrow parents\)\;
\While{\(|newPopulation| < N\)}{%
    Sample two parents, \(parentOne, parentTwo\), from \(parents\)\;
    \(offspring \longleftarrow Crossover(parentOne,parentTwo,C,P,)\)\;
    \(mutant \longleftarrow Mutation(offspring,\mu,R,C,P,w)\)\;
    \(newPopulation \longleftarrow newPopulation \cup \{mutant\}\)\;
}
\caption{\(CreateNewPopulation\)}
\ealg%


The statement of the EDO algorithm is deliberately vague here. This is, in part,
because of the customisable nature of its build but also to lay out its general
structure from a high level perspective. Lower level discussion is provided in
Section~\ref{subsection:mechanisms} where additional algorithms for the
individual creation, evolutionary operator and shrinkage processes are given.

Note that there are no defined processes for how to stop the algorithm or adjust
the mutation probability, \(p_m\). These form two key areas of potential
customisation since such conditions can be specific to the problem domain. With
that, as is detailed in the Python implementation, a user may define their
stopping condition or \(p_m\)-adjustment to be any reasonable function. Some
examples include:

\begin{itemize}
    \item These should be real-world examples used in other EA with references.
    \item Stopping when the difference in the variance of the current and last
        population fitnesses is below some tolerance.
    \item Halving the mutation probability every 100 iterations.
\end{itemize}

\subsection{Internal mechanisms}\label{subsection:mechanisms}

Below are detailed instructions and discussion around the internal mechanisms of
EDO.\ Each component has an affiliated algorithm statement and accompanying
diagram.

\subsubsection{Individuals}

Evolutionary algorithms operate on succeeding populations of individuals often
coined as ``generations''. Typically, a potential solution would be encoded as a
bit string of a fixed length and treated as a chromosome-like object to be
manipulated directly. In EDO, as the objective is to generate datasets, there is
no encoding or information reduction so that the datasets may be manipulated in
a more meaningful way with the biological operators.

In a sense, a dataset is treated similarly to a classical bit string as the
primary components (loci) of the dataset are considered to be its columns.

\inputtikz{individual.tex}{%
    An example of how an individual is first created.
}
\begin{tikzpicture}

    \path (-4, -10) pic {column=7}
          (1, -10) pic {column=7}
          (6, -10) pic {column=7};

    \node[ellipse, fill=orange!15] (dists) at (0, 1) {%
        \begin{tabular}{c}
            \tikz[baseline, inner sep=0] \node[anchor=base] (normal) {$N(\mu,
            \sigma^2)$}; \quad \quad \tikz[baseline, inner sep=0]
            \node[anchor=base] (uniform) {$U(\alpha, \beta)$}; \\
            {} \quad $Po(\lambda)$
        \end{tabular}
    } node[yshift=30pt, left=230pt] {%
        \scriptsize
        \begin{tabular}{l}
            Families of\\
            distributions
        \end{tabular}
    };

    \fill[cyan!15, rounded corners] node[yshift=-90pt,left=220pt] 
        {%
            \color{black}\scriptsize
            \begin{tabular}{l}
                Column\\
                information
            \end{tabular}
        }
        (-7.5,-4) rectangle (7.5,-2.5);
    
    \node (norm1) at (-5, -3.3) {$N(0.25, 1)$};
    \node (norm2) at (5, -3.3) {$N(-3.7, 0)$};
    \node (uniform1) at (0, -3.3) {$U(1.2, 3.2)$};

    \draw[->] ([xshift=-5pt] normal.south) to [out=270, in=90]
        ([yshift=1cm] norm1);
    \draw[->] ([yshift=-5pt] norm1.south) -- (-5, -5.5);

    \draw[->] ([xshift=0.1cm] normal.south) to [out=270, in=90]
        ([yshift=1cm] norm2) node[right=20pt, yshift=25pt] {%
            \color{blue}\scriptsize
            \begin{tabular}{l}
                Sample distribution\\
                and parameters
            \end{tabular}
        };
    \draw[->] ([yshift=-5pt] norm2.south) to (5, -5.5)
        node[right=20pt, yshift=15pt] {%
            \color{blue}\scriptsize
            \begin{tabular}{l}
                Sample values\\
                from distribution
            \end{tabular}
        };

    \draw[->] (uniform) to [out=270, in=90] ([yshift=1cm] uniform1);
    \draw[->] ([yshift=-5pt] uniform1.south) -- (0, -5.5);

    \draw[decorate, decoration={brace, amplitude=10pt}] (-7, -10) -- (-7, -6)
    node[midway, left=20pt] {%
        \scriptsize
        \begin{tabular}{l}
            Columns of\\
            the dataset
        \end{tabular}
    };

\end{tikzpicture}


\subsubsection{Selection}

\inputtikz{selection.tex}{%
    An example of the selection process with the inclusion of some lucky
    individuals.
}
\begin{tikzpicture}

    \fill[orange!15, rounded corners] (0, 0) rectangle (3, -6)
        node[below=90pt, midway] {\color{black} \scriptsize Old population};
    \fill[rounded corners, gray!5] (6, 0) rectangle (9, -6)
        node[below=90pt, midway] {\color{black} \scriptsize New population};

    \fill[pattern=north east lines, pattern color=orange]
        (0, -2) --
        ++(3, 0) {[rounded corners] --
        ++(0, 2) --
        ++(-3, 0)} --
        cycle
        {};
   
    \draw[->, dashed, orange, thick] (3.25, -1) to (5.75, -1);

    \fill[pattern=north east lines, pattern color=orange]
        (6, -2) --
        ++(3, 0) {[rounded corners] --
        ++(0, 2) --
        ++(-3, 0)} --
        cycle
        {};

    \foreach \val in {-5.5, -4, -3.2} {%
        \fill[pattern=north east lines, pattern color=magenta!85]
            (0, \val) rectangle (3, \val+0.1);
        \draw[->, dashed, magenta!85, thick]
            (3.25, \val) to [out=0, in=180] (5.75, -2.15);
    };

    \fill[pattern=north east lines, pattern color=magenta!85]
        (6, -2.3) rectangle (9, -2);

    \draw[decorate, decoration={brace, amplitude=10pt}] (-1, -2) -- (-1, 0)
        node[midway, left=20pt] {\scriptsize Best individuals};
    \draw[decorate, decoration={brace, amplitude=10pt}] (-1, -5.5) -- (-1, -3.1)
        node[midway, left=20pt] {\scriptsize Lucky individuals};
    \draw[decorate, decoration={brace, amplitude=10pt}] (10, 0) -- (10, -2.3)
        node[midway, right=20pt] {\scriptsize Parents};

    \node[%
        fill=blue!30, single arrow, minimum height=18mm, minimum width=4mm,
        single arrow head extend=3mm, anchor=north, rotate=-90]
        at (8, -3.8) {\tiny Crossover};

\end{tikzpicture}


\subsubsection{Crossover}

\inputtikz{crossover.tex}{An example of the crossover process.}
\subsection{Crossover}

Crossover is the operation of combining two individuals in order to create at
least one offspring. In EDO, a method known as uniform crossover is
used~\cite{Semenkin2012}. Under this method, a new individual is created by
uniformly sampling each of its
components from a set of two ``parent'' individuals. As can be seen in
Figure~\ref{figure:crossover}, this method has been adapted for the dataset
representation, i.e.\ two parent datasets have their dimensions and then columns
sampled uniformly and without replacement to give a new individual.

\inputtikz{crossover}{The crossover process.}
\subsection{Crossover}

Crossover is the operation of combining two individuals in order to create at
least one offspring. In EDO, a method known as uniform crossover is
used~\cite{Semenkin2012}. Under this method, a new individual is created by
uniformly sampling each of its
components from a set of two ``parent'' individuals. As can be seen in
Figure~\ref{figure:crossover}, this method has been adapted for the dataset
representation, i.e.\ two parent datasets have their dimensions and then columns
sampled uniformly and without replacement to give a new individual.

\inputtikz{crossover}{The crossover process.}
\subsection{Crossover}

Crossover is the operation of combining two individuals in order to create at
least one offspring. In EDO, a method known as uniform crossover is
used~\cite{Semenkin2012}. Under this method, a new individual is created by
uniformly sampling each of its
components from a set of two ``parent'' individuals. As can be seen in
Figure~\ref{figure:crossover}, this method has been adapted for the dataset
representation, i.e.\ two parent datasets have their dimensions and then columns
sampled uniformly and without replacement to give a new individual.

\inputtikz{crossover}{The crossover process.}
\input{tex/algorithms/crossover.tex}







\subsubsection{Mutation}

\inputtikz{mutation.tex}{An example of the mutation process.}
\balg%
\KwData{An individual, \(p_m\), \(R\), \(C\), \(\mathcal{P}\), \(w\)}
\KwResult{A mutated individual}

\Begin{%
    sample a random number \(r \in [0, 1]\)\;
    \If{\(r < p_m\) and adding a row would not violate \(R\)}{%
        sample a value from each distribution in the metadata\;
        append these values as a row to the end of the dataset\;
    }
    sample a new \(r \in [0, 1]\)\;
    \If{\(r < p_m\) and removing a row would not violate \(R\)}{%
        remove a row at random from the dataset
    }
    sample a new \(r \in [0, 1]\)\;
    \If{\(r < p_m\) and adding a new column would not violate \(C\)}{%
        create a new column using \(\mathcal{P}\) and \(w\)\;
        append this column to the end of the dataset
    }
    sample a new \(r \in [0, 1]\)\;
    \If{\(r < p_m\) and removing a column would not violate \(C\)}{%
        remove a column (and its associated metadata) at random from the dataset
    }
    \For{each distribution in the metadata}{%
        \For{each parameter of the distribution}{%
            sample a random number \(r \in [0, 1]\)\;
            \If{\(r < p_m\)}{%
                update the parameter value by sampling a new value from the
                associated distribution parameter limits
            }
        }
    }
    \For{each value in the dataset}{%
        sample a random number \(r \in [0, 1]\)\;
        \If{\(r < p_m\)}{%
            update the value by sampling a new value from the associated
            distribution in the metadata
        }
    }
}
\caption{The mutation process}
\ealg%


\subsubsection{Shrinking}

\inputtikz{shrinking.tex}{An diagram of the shrinking process.}
\balg%
\KwData{parents, current iteration, \(\mathcal{P}, M, s\)}
\KwResult{A new mutation space focussed around the parents}

\Begin{%
    calculate the shrink coefficient\;

    \For{distribution in \(\mathcal{P}\)}{%
        \For{each parameter of the distribution}{%
            get current values for parameter in parent columns\;
            find the mean value\;
            calculate the shift term using the values and shrink coefficient\;
            find the new lower and upper bounds around the mean\;
            set the parameter limits\;
        }
    }
}
\caption{Shrinking the mutation space}
\ealg%



\subsection{Implementation}\label{subsection:implementation}

\begin{itemize}
    \item Documentation: \url{https://edo.readthedocs.io}
    \item Repo: \url{https://github.com/daffidwilde/edo}
\end{itemize}

