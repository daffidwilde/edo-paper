\section{The evolutionary algorithm}\label{section:algorithm}

%---------------------
\subsection{Structure}

\begin{itemize}
    \item Algorithm statement with components
    \item Discussion on the choice to include custom stopping/dwindling
        conditions, compacting the search space, etc.
    \item Operators and detailed mechanisms to come later.
\end{itemize}

\balg%
\KwData{%
    \begin{tabular}{l}
    Fitness function, \(f: X \to \mathbb{R}\);
    population size, \(N\);
    row limits, \(R = [r_l, r_u]\);
    column limits, \(C = [c_l, c_u]\);\\

    Column distributions, \(P\);
    relative weights for \(P\), \(w \in {[0, 1]}^{|P|}\) s.t.\ \(\sum w = 1\);
    maximum iterations, \(M\);\\

    Best proportion, \(\delta\);
    lucky proportion, \(\epsilon\);
    mutation probability, \(\mu\);
    compaction ratio, \(s\).
    \end{tabular}
}
\KwResult{A full history of the populations and their fitnesses.}\;

\textcolor{blue}{initialisation}\;
\(population \longleftarrow CreateInitialPopulation(N,R,C,P,w)\)\;
\(populationFitness \longleftarrow GetPopulationFitness(population,f)\)\;
\(%
    populationHistory \longleftarrow%
    \left\{(population, populationFitness)\right\}
\)\;
\(i \longleftarrow 0\)\;\;

\textcolor{blue}{begin iterative step}\;
\While{\(iteration < M\) and not \textbf{STOP}}{%
    \textcolor{blue}{select parent individuals}\;
    \(%
        parents \longleftarrow%
        Selection(population,populationFitness,\delta,\epsilon)
    \)\;

    \textcolor{blue}{create new population}\;
    \(population \longleftarrow CreateNewPopulation(parents,N,R,C,P,w,\mu)\)\;

    \textcolor{blue}{update fitness and history}\;
    \(populationFitness \longleftarrow GetPopulationFitness(population,f)\)\;
    \(%
        populationHistory \longleftarrow%
        populationHistory \cup \left\{(population,populationFitness)\right\}
    \)\;

    \textcolor{blue}{adjust parameters}\;
    \(i \longleftarrow i + 1\)\;
    \(P \longleftarrow ReduceMutationSpace(parents,P,i,M,s)\)\;
    \(\mu, \textbf{STOP} \longleftarrow UpdateParameters\)\;
}
\caption{The evolutionary dataset optimisation algorithm}
\ealg%


\balg%
\KwData{\(N,R,C,P,w\)}
\KwResult{A random population of datasets of size \(N\), \(population\)}\;

\(population \longleftarrow \emptyset\)\;
\For{\(i\leftarrow 1 \ \KwTo \ N\)}{%
    \(%
        population \longleftarrow%
        population \cup \left\{CreateIndividual(R,C,P,w)\right\}
    \)\;
}
\caption{\(CreateInitialPopulation\)}
\ealg%

\balg%
\KwData{\(population,f\)}
\KwResult{Fitness of each individual in the population, \(populationFitness\)}\;

\(populationFitness \longleftarrow \emptyset\)\;
\For{\(individual \in population\)}{%
    \(%
        populationFitness \longleftarrow%
        populationFitness \cup \left\{f(individual)\right\}
    \)\;
}
\caption{\(GetPopulationFitness\)}
\ealg%

\balg%
\KwData{\(parents,N,R,C,P,w,\mu\)}
\KwResult{A new population of size \(N\), \(newPopulation\)}\;

\(newPopulation \longleftarrow parents\)\;
\While{\(|newPopulation| < N\)}{%
    Sample two parents, \(parentOne, parentTwo\), from \(parents\)\;
    \(offspring \longleftarrow Crossover(parentOne,parentTwo,C,P,)\)\;
    \(mutant \longleftarrow Mutation(offspring,\mu,R,C,P,w)\)\;
    \(newPopulation \longleftarrow newPopulation \cup \{mutant\}\)\;
}
\caption{\(CreateNewPopulation\)}
\ealg%

\balg%
\KwData{}
\KwResult{}\;

\textcolor{blue}{initialisation}\;
\(S \longleftarrow 1 - \frac{i}{s M}\)\;
\(currentParameters \longleftarrow GetCurrentParameters(parents, P)\)\;\;

\textcolor{blue}{compute new bounds on distribution parameters}\;
\For{\(parameters \in currentParameters\)}{%
    \For{\(parameter, parameterValues \in parameters\)}{%
        \(m \longleftarrow Mean(parameterValues)\)\;
        \(%
            \Delta \longleftarrow%
            \frac{S}{2} \times (\max parameterValues - \min parameterValues)
        \)
        \(%
            L \longleftarrow%
            \max\{parameterLimits[0], \min\{m - \Delta, m + \Delta\}\}
        \)\;
        \(%
            U \longleftarrow%
            \min\{parameterLimits[1], \max\{m - \Delta, m + \Delta\}\}
        \)\;
        \(parameterLimits \longleftarrow [L, U]\)\;
    }
}
\caption{\(ReduceMutationSpace\)}
\ealg%

\balg%
\(%
    currentParameters \longleftarrow%
    \left(p: \left(\emptyset, \ldots, \emptyset\right) | \ p \in P\right)
\)\;
\For{\(parent \in parents\)}{%
    \For{each column in \(parent\)}{%
        \(p \longleftarrow\) distribution class of column\;
        \If{no entry for \(p\) in \(currentParameters\)}{%
            \(currentParameters[p] \longleftarrow \emptyset\)
        }
        \For{\((parameter, parameterValue) \in p\)}{%
            \(%
                currentParameters[p] \longleftarrow%
                currentParameters[p] \cup%
                \left\{(parameter, parameterValue)\right\}
            \)\;
        }
    }
}
\caption{\(GetCurrentParameters\)}
\ealg%

Additional algorithms for the individual creation and operator processes are
given in Section~\ref{subsection:mechanisms}. However, there is no defined
process for how to stop the algorithm or adjust the mutation probability,
\(\mu\). This is deliberate as such conditions are specific to the problem
domain. As such, in the Python implementation, a user may define their stopping
condition or \(\mu\)-adjustment to be any function. Some examples include:

\begin{itemize}
    \item These should be real-world examples used in other EA with references.
    \item Stopping when the difference in the variance of the current and last
        population fitnesses is below some tolerance.
    \item Halving the mutation probability every 100 iterations.
\end{itemize}
        

%------------------------------------------------------------
\subsection{Internal mechanisms}\label{subsection:mechanisms}

\subsubsection{Individuals}

\inputtikz{individual.tex}{%
    An example of how an individual is first created.
}

\subsubsection{Selection}

\inputtikz{selection.tex}{%
    An example of the selection process with the inclusion of some lucky
    individuals.
}

\subsubsection{Crossover}

\inputtikz{crossover.tex}{An example of the crossover process.}

\subsubsection{Mutation}

\inputtikz{mutation.tex}{An example of the mutation process.}

\subsubsection{Shrinking}

\inputtikz{shrinking.tex}{An diagram of the shrinking process.}

%-----------------------------------------------------------
\subsection{Implementation}\label{subsection:implementation}

\begin{itemize}
    \item Documentation: \url{https://edo.readthedocs.io}
    \item Repo: \url{https://github.com/daffidwilde/edo}
\end{itemize}

