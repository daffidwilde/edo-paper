\section{The evolutionary algorithm}\label{section:algorithm}

\subsection{Structure}

\begin{itemize}
    \item Discussion on the choice to include custom stopping/dwindling
        conditions, compacting the search space, etc.
    \item Operators and detailed mechanisms to come later.
\end{itemize}

\balg%
\KwData{\(f, N, R, C, \mathcal{P}, w, M, b, l, \mu, s\)}
\KwResult{A full history of the populations and their fitnesses.}

\Begin{%
    create initial population of individuals\;
    find fitness of each individual\;
    record population and its fitness\;

    \While{%
        current iteration less than the maximum
        \textbf{and} stopping condition not met
    }{%
        select parents based on fitness and selection proportions\;
        use parents to create new population through crossover and mutation\;
        find fitness of each individual\;
        update population and fitness histories\;
        \If{adjusting the mutation probability}{%
            update mutation probability
        }
        \If{using a shrink factor}{%
            shrink the mutation space based on parents
        }
    }
}
\caption{The evolutionary dataset optimisation algorithm}
\ealg%
\label{alg:edo}
\balg%
\KwData{\(N,R,C,P,w\)}
\KwResult{A random population of datasets of size \(N\), \(population\)}\;

\(population \longleftarrow \emptyset\)\;
\For{\(i\leftarrow 1 \ \KwTo \ N\)}{%
    \(%
        population \longleftarrow%
        population \cup \left\{CreateIndividual(R,C,P,w)\right\}
    \)\;
}
\caption{\(CreateInitialPopulation\)}
\ealg%


\balg%
\KwData{parents, \(N, R, C, \mathcal{P}, w, p_m\)}
\KwResult{A new population of size \(N\)}

\Begin{%
    add parents to the new population\;
    \While{the size of the new population is less than \(N\)}{%
        sample two parents at random\;
        create an offspring by crossing over the two parents\;
        mutate the offspring according to the mutation probability\;
        add the mutated offspring to the population\;
    }
}
\caption{Creating a new population}
\ealg%


The statement of the EDO algorithm is deliberately vague here. This is, in part,
because of the customisable nature of its build but also to lay out its general
structure from a high level perspective. Lower level discussion is provided in
Section~\ref{subsection:mechanisms} where additional algorithms for the
individual creation, evolutionary operator and shrinkage processes are given.

Note that there are no defined processes for how to stop the algorithm or adjust
the mutation probability, \(p_m\). These form two key areas of potential
customisation since such conditions can be specific to the problem domain. With
that, as is detailed in the Python implementation, a user may define their
stopping condition or \(p_m\)-adjustment to be any reasonable function. Some
examples include:

\begin{itemize}
    \item These should be real-world examples used in other EA with references.
    \item Stopping when the difference in the variance of the current and last
        population fitnesses is below some tolerance.
    \item Halving the mutation probability every 100 iterations.
\end{itemize}

\subsection{Internal mechanisms}\label{subsection:mechanisms}

Below are detailed instructions and discussion around the internal mechanisms of
EDO.\ Each component has an affiliated algorithm statement and accompanying
diagram.

\subsubsection{Individuals}

Evolutionary algorithms operate on succeeding populations of individuals often
coined as ``generations''. Typically, a potential solution would be encoded as a
bit string of a fixed length and treated as a chromosome-like object to be
manipulated directly. In EDO, as the objective is to generate datasets, there is
no encoding or information reduction so that the datasets may be manipulated in
a more meaningful way with the biological operators.

In a sense, a dataset is treated similarly to a classical bit string as the
primary components (loci) of the dataset are considered to be its columns.

\inputtikz{individual.tex}{%
    An example of how an individual is first created.
}
\balg%
\KwData{\(R, C, \mathcal{P}, w\)}
\KwResult{An individual defined by a dataset and some metadata}

\Begin{%
    sample a number of rows and columns\;
    create an empty dataset\;
    \For{each column in the dataset}{%
        sample a distribution from \(\mathcal{P}\)\;
        create an instance of the distribution\;
        fill in the column by sampling from this instance\;
        record the instance in the metadata
    }
}
\caption{Creating an individual}
\ealg%


\subsubsection{Selection}

\inputtikz{selection.tex}{%
    An example of the selection process with the inclusion of some lucky
    individuals.
}
\balg%
\KwIn{population, population fitness, \(b\), \(l\)}
\KwOut{A set of parent individuals}

\Begin{%
    calculate \(n_b\) and \(n_l\)\;
    sort the population by the fitness of its individuals\;
    take the first \(n_b\) individuals and make them parents\;
    \If{there are any individuals left}{%
        take the next \(n_l\) individuals and make them parents\;
    }
}
\caption{The selection process}\label{alg:selection}
\ealg%


\subsubsection{Crossover}

\inputtikz{crossover.tex}{An example of the crossover process.}
\balg%
\KwIn{Two parents}
\KwOut{An offspring made from the parents ready for mutation}

\Begin{%
    collate the columns and metadata from each parent in a pool\;
    sample each dimension from between the parents uniformly\;
    form an empty dataset with these dimensions\;
    \For{each column in the dataset}{%
        sample a column (and its corresponding metadata) from the pool\;
        \If{this column is longer than required}{%
            randomly select entries and delete them as needed 
        }
        \If{this column is shorter than required}{%
            sample new values from the metadata and append them to the column as
            needed
        }
        add this column to the dataset and record its metadata\;
    }
}
\caption{The crossover process}
\ealg%


\subsubsection{Mutation}

\inputtikz{mutation.tex}{An example of the mutation process.}
\balg%
\KwData{An individual, \(p_m\), \(R\), \(C\), \(\mathcal{P}\), \(w\)}
\KwResult{A mutated individual}

\Begin{%
    sample a random number \(r \in [0, 1]\)\;
    \If{\(r < p_m\) and adding a row would not violate \(R\)}{%
        sample a value from each distribution in the metadata\;
        append these values as a row to the end of the dataset\;
    }
    sample a new \(r \in [0, 1]\)\;
    \If{\(r < p_m\) and removing a row would not violate \(R\)}{%
        remove a row at random from the dataset
    }
    sample a new \(r \in [0, 1]\)\;
    \If{\(r < p_m\) and adding a new column would not violate \(C\)}{%
        create a new column using \(\mathcal{P}\) and \(w\)\;
        append this column to the end of the dataset
    }
    sample a new \(r \in [0, 1]\)\;
    \If{\(r < p_m\) and removing a column would not violate \(C\)}{%
        remove a column (and its associated metadata) at random from the dataset
    }
    \For{each distribution in the metadata}{%
        \For{each parameter of the distribution}{%
            sample a random number \(r \in [0, 1]\)\;
            \If{\(r < p_m\)}{%
                sample a new value from within the distribution parameter
                limits\;
                update the parameter value with this new value
            }
        }
    }
    \For{each entry in the dataset}{%
        sample a random number \(r \in [0, 1]\)\;
        \If{\(r < p_m\)}{%
            sample a new value from the associated column distribution\;
            update the entry with this new value
        }
    }
}
\caption{The mutation process}\label{algorithm:mutation}
\ealg%


\subsubsection{Shrinking}

\inputtikz{shrinking.tex}{An diagram of the shrinking process.}
\balg%
\KwIn{parents, current iteration, \(\mathcal{P}, M, s\)}
\KwOut{A new mutation space focussed around the parents}

\Begin{%
    \For{each distribution subtype in \(\mathcal{P}\)}{%
        \For{each parameter of the distribution}{%
            get the current values for parameter over all parent columns\;
            find the mean of the current values\;
            find the new lower~(\ref{eq:shrinking_lower}) and
            upper~(\ref{eq:shrinking_upper}) bounds around the mean\;
            set the parameter limits\;
        }
    }
}
\caption{Shrinking the mutation space}\label{alg:shrinking}
\ealg%



\subsection{Implementation}\label{subsection:implementation}

\begin{itemize}
    \item Documentation: \url{https://edo.readthedocs.io}
    \item Repo: \url{https://github.com/daffidwilde/edo}
\end{itemize}

