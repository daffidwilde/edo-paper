\subsection{Crossover}

Crossover is the operation of combining two individuals in order to create at
least one offspring. In genetic algorithms, the term `crossover' can be taken
literally: two bit strings are crossed at a point to create two new bit strings.
Another popular method is uniform crossover, which has been favoured for its
efficiency and efficacy in combining individuals~\cite{Semenkin2012}. For EDO,
this method is adapted to support dataset manipulation: a new individual is
created by uniformly sampling each of its components (dimensions and then
columns) from a set of two `parent' individuals, as shown in
Figure~\ref{fig:crossover}.

\inputtikz[\linewidth]{crossover}{%
    The crossover process between two individuals with different dimensions.
}

Observe that there is no requirement on the dimensions of the parents to be of
similar or equal shapes. This is because the driving aim of the proposed method
is to explore the space of all possible datasets. In the case where there is
incongruence in the lengths of the two parents, missing values may appear in a
shorter column that is sampled. To resolve this, values are sampled from the
probability distribution associated with that column to fill in these gaps.

\balg%
\KwIn{Two parents}
\KwOut{An offspring made from the parents ready for mutation}

\Begin{%
    collate the columns and metadata from each parent in a pool\;
    sample each dimension from between the parents uniformly\;
    form an empty dataset with these dimensions\;
    \For{each column in the dataset}{%
        sample a column (and its corresponding metadata) from the pool\;
        \If{this column is longer than required}{%
            randomly select entries and delete them as needed 
        }
        \If{this column is shorter than required}{%
            sample new values from the metadata and append them to the column as
            needed
        }
        add this column to the dataset and record its metadata\;
    }
}
\caption{The crossover process}
\ealg%


