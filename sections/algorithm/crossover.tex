\subsection{Crossover}

Crossover is the operation of combining two individuals in order to create at
least one offspring. In genetic algorithms, the term ``crossover'' can be taken
literally: two bit strings are crossed at a point to create two new bit strings.
Another popular method is uniform crossover, which has been favoured for its
efficiency and efficacy in combining individuals~\cite{Semenkin2012}. For EDO,
this method is adapted to support dataset manipulation: a new individual is
created by uniformly sampling each of its components (dimensions and then
columns) from a set of two ``parent'' individuals, as shown in
Figure~\ref{figure:crossover}.

\inputtikz{crossover}{%
    The crossover process between two individuals with different dimensions.
}
\subsection{Crossover}

Crossover is the operation of combining two individuals in order to create at
least one offspring. In EDO, a method known as uniform crossover is
used~\cite{Semenkin2012}. Under this method, a new individual is created by
uniformly sampling each of its
components from a set of two ``parent'' individuals. As can be seen in
Figure~\ref{figure:crossover}, this method has been adapted for the dataset
representation, i.e.\ two parent datasets have their dimensions and then columns
sampled uniformly and without replacement to give a new individual.

\inputtikz{crossover}{The crossover process.}
\subsection{Crossover}

Crossover is the operation of combining two individuals in order to create at
least one offspring. In EDO, a method known as uniform crossover is
used~\cite{Semenkin2012}. Under this method, a new individual is created by
uniformly sampling each of its
components from a set of two ``parent'' individuals. As can be seen in
Figure~\ref{figure:crossover}, this method has been adapted for the dataset
representation, i.e.\ two parent datasets have their dimensions and then columns
sampled uniformly and without replacement to give a new individual.

\inputtikz{crossover}{The crossover process.}
\subsection{Crossover}

Crossover is the operation of combining two individuals in order to create at
least one offspring. In EDO, a method known as uniform crossover is
used~\cite{Semenkin2012}. Under this method, a new individual is created by
uniformly sampling each of its
components from a set of two ``parent'' individuals. As can be seen in
Figure~\ref{figure:crossover}, this method has been adapted for the dataset
representation, i.e.\ two parent datasets have their dimensions and then columns
sampled uniformly and without replacement to give a new individual.

\inputtikz{crossover}{The crossover process.}
\input{tex/algorithms/crossover.tex}







