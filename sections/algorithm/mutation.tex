\subsection{Mutation}\label{subsection:mutation}

Mutation is used in evolutionary algorithms to encourage a broader exploration
of the search space at each generation. Under this framework, the mutation
process manipulates the phenotype of an individual where numerous things need to
be modified including an individual's dimensions, column metadata and the
entries themselves. This process is described in Figure~\ref{fig:mutation}.

\inputtikz[\linewidth]{mutation}{The mutation process.}

As shown in Figure~\ref{fig:mutation}, each of the potential mutations occur
with the same probability, \(p_m\). However, the way in which columns are
maintained assure that (assuming appropriate choices for \(f\) and
\(\mathcal{P}\)) many mutations in the metadata and the dataset itself will only
result in some incremental change in the individual's fitness
relative to, say, a completely new individual.

\balg%
\KwData{An individual, \(p_m\), \(R\), \(C\), \(\mathcal{P}\), \(w\)}
\KwResult{A mutated individual}

\Begin{%
    sample a random number \(r \in [0, 1]\)\;
    \If{\(r < p_m\) and adding a row would not violate \(R\)}{%
        sample a value from each distribution in the metadata\;
        append these values as a row to the end of the dataset\;
    }
    sample a new \(r \in [0, 1]\)\;
    \If{\(r < p_m\) and removing a row would not violate \(R\)}{%
        remove a row at random from the dataset
    }
    sample a new \(r \in [0, 1]\)\;
    \If{\(r < p_m\) and adding a new column would not violate \(C\)}{%
        create a new column using \(\mathcal{P}\) and \(w\)\;
        append this column to the end of the dataset
    }
    sample a new \(r \in [0, 1]\)\;
    \If{\(r < p_m\) and removing a column would not violate \(C\)}{%
        remove a column (and its associated metadata) at random from the dataset
    }
    \For{each distribution in the metadata}{%
        \For{each parameter of the distribution}{%
            sample a random number \(r \in [0, 1]\)\;
            \If{\(r < p_m\)}{%
                sample a new value from within the distribution parameter
                limits\;
                update the parameter value with this new value
            }
        }
    }
    \For{each entry in the dataset}{%
        sample a random number \(r \in [0, 1]\)\;
        \If{\(r < p_m\)}{%
            sample a new value from the associated column distribution\;
            update the entry with this new value
        }
    }
}
\caption{The mutation process}\label{algorithm:mutation}
\ealg%

