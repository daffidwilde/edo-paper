\subsection{Selection}

The selection operator describes the process by which individuals are chosen
from the current population to generate the next. Almost always, the likelihood
of an individual being selected is determined by their fitness. This is because
the purpose of selection is to preserve favourable qualities and encourage some
homogeneity within future generations~\cite{Back1994}.

\inputtikz[.8\imgwidth]{selection}{%
    The selection process with the inclusion of some lucky individuals.
}
\balg%
\KwIn{population, population fitness, \(b\), \(l\)}
\KwOut{A set of parent individuals}

\Begin{%
    calculate \(n_b\) and \(n_l\)\;
    sort the population by the fitness of its individuals\;
    take the first \(n_b\) individuals and make them parents\;
    \If{there are any individuals left}{%
        take the next \(n_l\) individuals and make them parents\;
    }
}
\caption{The selection process}\label{alg:selection}
\ealg%


In EDO, a modified truncation selection method is used~\cite{Jebari2013}, as can
be seen in Figure~\ref{figure:selection}. Truncation selection takes a fixed
number, \(n_b = \lceil bN\rceil\), of the fittest individuals in a population
and makes them the ``parents'' of the next. It has been observed that, despite
its efficiency as a selection operator, truncation selection can lead to
premature convergence at local optima~\cite{Jebari2013, Tatsuya2002}. The
modification for EDO is an optional stage after the best individuals have been
chosen: with some small \(l\), a number, \(n_l = \lceil lN\rceil\), of the
remaining individuals can be selected at random to be carried forward. Hence,
allowing for a small number of randomly selected individuals may encourage
diversity and further exploration throughout the run of the algorithm. It should
be noted that regardless of this step, an individual could potentially be
present throughout the entirety of the algorithm.

After the parents have been selected, there are two adjustments made to the
current search space. The first is that the subtypes for each family in
\(\mathcal{P}\) are updated to only those present in the parents. The second
adjustment is a process which acts on the distribution parameter limits for
each subtype in \(\mathcal{P}\). This adjustment gives the ability to ``shrink''
the search space about the region observed in a given population. This method is
based on a power law described in~\cite{Amirjanov2016} that relies on a shrink
factor, \(s\). At each iteration, \(t\), every distribution subtype which is
present in the parents has its parameter's limits, \(\left(l_t, u_t\right)\),
adjusted. This adjustment is such that the new limits, \(\left(l_{t+1},
u_{t+1}\right)\) are centred about the mean observed value, \(\mu\), for that
parameter:
\begin{align}
    \label{eq:shrinking_lower}
    l_{t+1}&= \max \left\{l_t, \ \mu - \frac{1}{2} (u_t - l_t) s^t\right\}\\
    \label{eq:shrinking_upper}
    u_{t+1}&= \min \left\{u_t, \ \mu + \frac{1}{2} (u_t - l_t) s^t\right\}
\end{align}

The shrinking process is given explicitly in
Algorithm~\ref{algorithm:shrinking}. Note that the behaviour of this process can 
produce reductive results for some use cases and is optional.

\balg%
\KwIn{parents, current iteration, \(\mathcal{P}, M, s\)}
\KwOut{A new mutation space focussed around the parents}

\Begin{%
    \For{each distribution subtype in \(\mathcal{P}\)}{%
        \For{each parameter of the distribution}{%
            get the current values for parameter over all parent columns\;
            find the mean of the current values\;
            find the new lower~(\ref{eq:shrinking_lower}) and
            upper~(\ref{eq:shrinking_upper}) bounds around the mean\;
            set the parameter limits\;
        }
    }
}
\caption{Shrinking the mutation space}\label{alg:shrinking}
\ealg%


