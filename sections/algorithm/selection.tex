\subsection{Selection}

The selection operator describes the process by which individuals are chosen
from the current population to generate the next. Almost always, the likelihood
of an individual being selected is determined by their fitness. This is because
the purpose of selection is to preserve favourable qualities and encourage some
homogeneity within future generations~\cite{Back1994}.

\inputtikz[.8\textwidth]{selection}{%
    The selection process with the inclusion of some lucky individuals.
}
\begin{tikzpicture}

    \fill[orange!15, rounded corners] (0, 0) rectangle (3, -6)
        node[below=90pt, midway] {\color{black} \scriptsize Old population};
    \fill[rounded corners, gray!5] (6, 0) rectangle (9, -6)
        node[below=90pt, midway] {\color{black} \scriptsize New population};

    \fill[pattern=north east lines, pattern color=orange]
        (0, -2) --
        ++(3, 0) {[rounded corners] --
        ++(0, 2) --
        ++(-3, 0)} --
        cycle
        {};
   
    \draw[->, dashed, orange, thick] (3.25, -1) to (5.75, -1);

    \fill[pattern=north east lines, pattern color=orange]
        (6, -2) --
        ++(3, 0) {[rounded corners] --
        ++(0, 2) --
        ++(-3, 0)} --
        cycle
        {};

    \foreach \val in {-5.5, -4, -3.2} {%
        \fill[pattern=north east lines, pattern color=magenta!85]
            (0, \val) rectangle (3, \val+0.1);
        \draw[->, dashed, magenta!85, thick]
            (3.25, \val) to [out=0, in=180] (5.75, -2.15);
    };

    \fill[pattern=north east lines, pattern color=magenta!85]
        (6, -2.3) rectangle (9, -2);

    \draw[decorate, decoration={brace, amplitude=10pt}] (-1, -2) -- (-1, 0)
        node[midway, left=20pt] {\scriptsize Best individuals};
    \draw[decorate, decoration={brace, amplitude=10pt}] (-1, -5.5) -- (-1, -3.1)
        node[midway, left=20pt] {\scriptsize Lucky individuals};
    \draw[decorate, decoration={brace, amplitude=10pt}] (10, 0) -- (10, -2.3)
        node[midway, right=20pt] {\scriptsize Parents};

    \node[%
        fill=blue!30, single arrow, minimum height=18mm, minimum width=4mm,
        single arrow head extend=3mm, anchor=north, rotate=-90]
        at (8, -3.8) {\tiny Crossover};

\end{tikzpicture}


In EDO, a modified truncation selection method is used, as can be seen in
Figure~\ref{figure:selection}. Truncation selection takes a fixed number, \(n_b
= \lceil bN\rceil\), of the fittest individuals in a population and makes them
the ``parents'' of the next. The modification is an optional stage after the
best individuals have been chosen: by passing some small \(l\) to the
evolutionary algorithm, a number of the remaining individuals can be selected at
random to be carried forward. This number is given by \(n_l = \lceil lN
\rceil\). It should be noted that even with this modification, no individual may
be selected more than once in a single iteration but could potentially be
present throughout the run of the algorithm.

It has been observed that, despite its efficiency as a selection operator,
truncation selection can lead to premature convergence at local
optima~\cite{Jebari2013, Tatsuya2002}. Hence, allowing for the inclusion of a
small number of randomly selected individuals may encourage diversity and
exploration throughout the run of the algorithm.

A final component to this part of the evolutionary algorithm provides the
ability to ``shrink'' the search space about the region observed in a given
population. This method is based on a power law described
in~\cite{Amirjanov2016} that relies on a shrink factor, \(s\). At each
iteration, \(t\), the parents are taken and every present distribution's
parameter limits, \(\left(l_t, u_t\right)\), are centred about the mean parent
value, \(\mu\):
\begin{align}
    \label{eq:shrinking_lower}
    l_{t+1}&= \max \left\{l_t, \ \mu - \frac{1}{2} (u_t - l_t) s^t\right\}\\
    \label{eq:shrinking_upper}
    u_{t+1}&= \max \left\{u_t, \ \mu + \frac{1}{2} (u_t - l_t) s^t\right\}
\end{align}

This shrinking process is described in Algorithm~\ref{algorithm:shrinking} but
note that the behaviour of this process can give reductive results for some use
cases. For instance, where there exist multiple optima for dataset
characteristics.

\balg%
\KwData{parents, current iteration, \(\mathcal{P}, M, s\)}
\KwResult{A new mutation space focussed around the parents}

\Begin{%
    calculate the shrink coefficient\;

    \For{distribution in \(\mathcal{P}\)}{%
        \For{each parameter of the distribution}{%
            get current values for parameter in parent columns\;
            find the mean value\;
            calculate the shift term using the values and shrink coefficient\;
            find the new lower and upper bounds around the mean\;
            set the parameter limits\;
        }
    }
}
\caption{Shrinking the mutation space}
\ealg%


