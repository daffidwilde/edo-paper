\subsection{Individuals}

Evolutionary algorithms operate in an iterative process on populations of
individuals that each represent a solution to the problem in question. In a
genetic algorithm, an individual is a solution encoded as a bit string of,
typically, fixed length and treated as a chromosome-like object to be
manipulated. In EDO, as the objective is to generate datasets and explore the
space in which datasets exist, there is no encoding. As such the distinction is
made that EDO is an evolutionary algorithm. 

As is seen in Figure~\ref{figure:individual}, an individual's creation is
defined by the generation of its columns. A set of instructions on how to sample
new values (in mutation, for instance, Section~\ref{subsection:mutation}) for
that column are recorded in the form of a probability distribution. These
distributions are sampled and created from the families passed in
\(\mathcal{P}\). In EDO, the produced datasets and their metadata are
manipulated directly so that the biological operators can be designed and be
interpreted in a more meaningful way as will be seen later in this section.

However, one should not assume that the columns are a reliable representative of
the distribution associated with them, or vice versa. This is particularly true
of ``shorter'' datasets with a small number of rows, whereas confidence in the
pair could be given more liberally for ``longer'' datasets with a larger number
of rows. In any case, appropriate methods should be

\inputtikz{individual}{%
    An example of how an individual is first created.
}
\balg%
\KwData{\(R, C, \mathcal{P}, w\)}
\KwResult{An individual defined by a dataset and some metadata}

\Begin{%
    sample a number of rows and columns\;
    create an empty dataset\;
    \For{each column in the dataset}{%
        sample a distribution from \(\mathcal{P}\)\;
        create an instance of the distribution\;
        fill in the column by sampling from this instance\;
        record the instance in the metadata
    }
}
\caption{Creating an individual}
\ealg%


