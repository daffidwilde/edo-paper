\appendix
\section{Appendix}

\subsection{Lloyd's algorithm}\label{appendix:kmeans}

\balg%
\KwIn{a dataset \(X\), a number of centroids \(k\), a distance metric \(d\)}
\KwOut{a partition of \(X\) into \(k\) parts, \(Z\)}

\Begin{%
    select \(k\) initial centroids, \(z_1, \ldots, z_k \in X\)\;
    \While{any point changes cluster or some stopping criterion is not met}{%
        assign each point, \(x \in X\), to cluster \(Z_{j^*}\) where:
        \[
            j^* = \argmin_{j = 1, \ldots, k} \left\{%
                {d\left(x, z_j\right)}^2
            \right\}
        \]\;
        recalculate all centroids by taking the intra-cluster mean:
        \[
            z_j = \frac{1}{|Z_j|} \sum_{x \in Z_j} x
        \]
    }
}
\caption{\(k\)-means (Lloyd's)}\label{algorithm:kmeans}
\ealg%


\subsection{Implementation example}\label{appendix:code}

Below is an example of how the Python implementation was used to complete the
first example, including the definition of the fitness function.

\singlespacing\begin{minted}{python}
import edo
from edo.pdfs import Uniform
from sklearn.cluster import KMeans


def fitness(dataframe, seed):
    """ Return the final inertia of 2-means on dataframe. """

    km = KMeans(n_clusters=2, random_state=seed).fit(dataframe)
    return km.inertia_


Uniform.param_limits["bounds"] = [0, 1]
row_limits, col_limits = [3, 100], [2, 2]

edo.run_algorithm(
    fitness,
    size,
    row_limits,
    col_limits,
    families=[Uniform],
    max_iter=1000,
    best_prop=selection,
    mutation_prob=mutation,
    seed=seed,
    root="out",
    fitness_kwargs={"seed": seed},
)
\end{minted}
